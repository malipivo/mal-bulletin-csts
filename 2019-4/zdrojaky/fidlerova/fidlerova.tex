% !TEX TS-program = LuaLaTeX
% !TEX encoding = UTF-8 Unicode 
% !TEX root = ../../!hlavni-soubor.tex
\def\zvyrazni#1{\textit{#1}}

\gdef\mujnazevCS{Spomienky na STAKAN 2019}
\nazev{\mujnazevCS}

\gdef\mujnazevEN{Memories of STAKAN 2019}
\nazev{\mujnazevEN}

\gdef\mujnazevPR{\mujnazevCS}
\gdef\mujnazevDR{\mujnazevEN}

\gdef\mujauthor{Helena Fidlerová}
\author{\mujauthor}

%\Adresa{}

\Email{helena.fidlerova@stuba.sk}

%\Abstrakt{}
%\smallskip
%\KlicovaSlova{}

%\Abstract{}
%\smallskip
%\KeyWords{}

\medskip

\noindent
Rozhodnutie, prečo ísť na konferenciu STAKAN organizovanú 11.\,--\,13. októbra 2019 Českou štatistickou spoločnosťou (ČStS) a~Slovenskou štatistickou a~demografickou spoločnosťou (SŠDS), bolo veľmi spontánne a~jednoznačné, keď som sa dozvedela, že tohtoročnou témou konferencie bude výuka štatistiky pre nematematické odbory na stredných a~vysokých školách. Počet odborných konferencií z~oblasti štatistiky a~pravdepodobnosti je veľký, ale toto bolo to, čo som už dlho hľadala.  Výučbu štatistických metód na ústave priemyselného inžinierstva a~manažmentu MTF STU Trnava  zabezpečujem na našom pracovisku s prestávkami od roku 2003. Dúfala som, že nájdem odpovede na moje otázky, ako učiť milovanú i~nenávidenú štatistiku.

Seminár sa konal v krásnom prostredí zámku Křtiny, kam to bolo zo všetkých kútov Slovenska i Česka blízko. Odborný seminár bol naplnený  príspevkami širokého spektra odborníkov z~oblasti teórie, výuky a uplatnenia štatistiky, predovšetkým z českých a~slovenských vysokých škôl.

V úvode nás srdečne privítali predseda ČStS  Ondřej Vencálek a~predsedníčka SŠDS doc. Iveta Stankovičová, a hneď som vedela, že som na správnom mieste. Pozvané prednášky boli plné charizmatických osobností ako jsou profesor Ryozo Miura z Hitotsubashi University s témou \zvyrazni{My experience of Teach\-ing Statistics with a~software JMP at Hitotsubashi University} a profesor Jakub Fischer z~Vysokej školy Ekonomickej v~Prahe s témou \zvyrazni{Ekonomická a~matematická statistika -- vzájemné p(r)otkávání nejen ve výuce.} Následne odzneli i~skúsenosti s výučbou na rôznych typoch škôl v Čechách, na Slovensku, ale i ďalekom Ománe (J. Mačutek, I.  Stankovičová, Z. Šulc, M. Zikmundová a ďalší). Do pekelnej reality nás vniesol Tomáš Fürst s jeho vystúpením na tému \zvyrazni{Perverzní incentivy a~špatná statistika: Pohled z pekla.}

Výnimočné pre mňa bolo i osobné stretnutie s ľuďmi, ktorých publikácie boli pre mňa prvou barličkou pri poznávaní tajov štatistiky, ako pani prof. Ing. Hana Řezanková, CSc. Viaceré príspevky mali  špecializovaný štatisticky obsah (P. Martinková, H. Šimková, T. Hajdúková, I. Waczulíková, J. Antoch). Skloňovala sa i Hejného metóda v príspevku Ota Přibylu a Pavla Hejného.  

Široko bola diskutovaná problematika i úskalia používania rôznych štatistických softwarov pri výuke štatistiky, či už platených verzií alebo freewarov. Zhoda panovala v názore, že študenti rôzneho zamerania sa musia zamerať na pochopenie, výber vhodnej metódy, aplikovanie a najmä správnu interpretáciu výsledkov.  

Účelom ani cieľom nie je spomenúť všetkých autorov i~príspevky, ktoré odzneli, tak len skonštatujem, že to stálo za to a niektoré príspevky nájdete publikované v Informačnom bulletine Českej štatistickej spoločnosti. Diskusia prebiehala vo veľmi priateľskej a srdečnej atmosfére, akú som doteraz nikde inde nezažila\ldots{} Ďakujem za to všetkým zúčastneným a predovšetkým usporiadateľom. Konštruktívne a zanietené rozhovory na vážne i nevážne témy pokračovali i po oficiálnom konci programu konferencie. Padol i návrh na nové logo STAKAN, ktoré verím, že sa stane skutočnosťou. Zúčastnení vedia a ostatní nech sa nechajú prekvapiť. Príjemným prekvapením od organizátorov konferencie bola návšteva kvapľovej jaskyne Výpustek bez kvapľov.

Záverom môžem skonštatovať, že som sa cítila ako v rodine, medzi svojimi  i keď som bola na tomto podujatí prvýkrát. Určite však nie posledný raz. Našla som viaceré inšpirácie i~odpovede na mnohé otázky, ale stále som na ceste k~objavovaniu. Záverom len konštatovanie smerom k štatistickej analýze. Keď ju miluješ, nie je čo riešiť.

\vfil

\parindent=0pt

\newdimen\maldelka \maldelka=\textwidth
\advance\maldelka by -2pt
\fboxrule=1pt
\fboxsep=0pt

\fbox{\includegraphics[trim=0pt 20mm 0pt 0pt, clip, width=\maldelka]{12.JPG}}

\hfil Účastníci konference STAKAN 2019 -- Křtiny


\newpage
\fbox{\includegraphics[trim=0pt 20mm 0pt 0pt, clip, width=\maldelka]{09.JPG}}

\hfil Posluchači při úvodní přednášce Jakuba Fischera

\vfill

\fbox{\includegraphics[trim=0pt 20mm 0pt 0pt, clip, width=\maldelka]{02.JPG}}

\hfil Halina Šimková mluví o forenzní analýze DNA


\newpage

\fbox{\includegraphics[trim=0pt 20mm 0pt 0pt, clip, width=\maldelka]{11.JPG}}

\hfil Účastníci konference naslouchají křtinské zvonkohře

\vfill

\fbox{\includegraphics[trim=0pt 20mm 0pt 0pt, clip, width=\maldelka]{01.JPG}}

\hfil Exkurze v jeskyni Výpustek


\newpage

\fbox{\includegraphics[trim=0pt 20mm 0pt 0pt, clip, width=\maldelka]{05.JPG}}

\hfil Momentka z konference

\vfill

\fbox{\includegraphics[trim=0pt 0mm 0pt 20mm, clip, width=\maldelka]{04.JPG}}

\hfil Místo konání -- zámek Křtiny a přilehlý barokní kostel Jména Panny Marie



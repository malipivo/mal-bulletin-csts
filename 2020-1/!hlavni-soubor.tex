\listfiles
%\nonstopmode

\newcount\zahlavi
\zahlavi=0 % 0 či 1, dát červenou pracovní verze?
%\IfFileExists{zahlavi.tex}{\input zahlavi.tex}{}

% !TEX TS-program = lualatex
% !TEX encoding = UTF-8
\documentclass[a4paper,10pt,twoside]{article}
\usepackage[czech,english,ngerman]{babel}
% german,germanb,slovak,
\usepackage{csbulacronym}

\usepackage[utf8]{luainputenc}
\usepackage[IL2]{fontenc}
\usepackage{amssymb,amsmath,amsfonts, amsbsy,amsthm}
%\newtheorem{theorem}{Theorem}
\usepackage[final]{graphicx} % [draft]
%\usepackage{epstopdf}
%\usepackage{stmaryrd}
%\usepackage{fontspec}
%\usepackage{luaotfload}
%\usepackage{fontspec}

  %\usepackage{algorithmic}
  %\usepackage{algorithm}
  %\usepackage{tipa}
  %\let\oldtipa=\tipa
  %\def\tipa#1{\oldtipa{#1}}
  %\usepackage[figuresright]{rotating}
  %\usepackage{multirow}
  
\usepackage{xcolor}
\usepackage[figuresright]{rotating}
%\usepackage{zref-savepos}
%\newcommand*{\currpos}[1]{%
  %\zsavepos{#1}%
  %%(\zposx{#1}sp, \zposy{#1}sp) =
  %%(\the\dimexpr\zposx{#1}sp\relax, \the\dimexpr\zposy{#1}sp\relax)%
%}

%\def\cellcolor#1{}
%\usepackage{psfrag, afterpage, enumerate, longtable, colortbl, } %
\usepackage{afterpage}
\usepackage{tabularx}
%\usepackage{longtable}

\usepackage{multirow}
\usepackage{colortbl}
%\usepackage{multirow,colortbl,latexsym}
\usepackage{tikz}
\usetikzlibrary{calc}
\usepackage{zref-savepos}
%\usepackage{tabu}
\usepackage{hhline}
%\usetikzlibrary{positioning}
%\usepackage{smejliky}
%\usetikzlibrary{positioning, shapes.geometric, calc}
%\usetikzlibrary{arrows.meta, calc}
%\usepackage{pgfplots}
%\pgfplotsset{compat=1.11}
%\pgfplotsset{width=95mm,compat=1.10}
\usepackage{wrapfig}
%\usepackage{cite}
\def\citepunct{,\,}
\def\prostor{}
%\input glyphtounicode
%\pdfgentounicode=1
%\usepackage{auto-pst-pdf}

%\usepackage{makeidx}
%\makeindex
%\usepackage{multicol}
%\usepackage[flushleft]{threeparttable}
%\usepackage{longtable}

% stříž
%\usepackage{pgfplots,pgfplotstable,animate}
%\pgfplotsset{width=1\textwidth, height=3cm}
%\usepackage{verbatim,wasysym}


\headheight=12.5pt
\headsep=17pt % 19pt
\usepackage{fancyhdr}
\pagestyle{fancy}
\fancyhf{}
\fancyfoot[LE,RO]{\thepage}
\fancyheadoffset[LE,RO]{0pt}
% \marginparsep+\marginparwidth
\renewcommand{\headrulewidth}{0.4pt}
\renewcommand{\footrulewidth}{0.4pt}

\def\pracovnizahlavi{ {\color{red}\bf pracovní verze}}
\fancyhead[RO]{Informační bulletin České statistické společnosti, 
  %!\the\cislo
  \cislalong
  /\pgfmathparse{int(\rocnik+2008)}\pgfmathresult
\ifnum\zahlavi=1
  \pracovnizahlavi\relax
\else
  \relax
\fi
%\ \pracovnizahlavi % NEW!
}
%\fancyfoot[RE]{\ifnum\zahlavi=0\pracovnizahlavi\fi}
\fancyhead[LE]{\dodatek}

\def\formatujosnovu#1{\textbf{#1}\smallskip\par}
\def\zasah#1{%
  \def\dodatek{#1}%
  \write\maosnova{\formatujosnovu{#1}}%
  \write\maosnovab{$<$tr$>$$<$th colspan=$"$4$"$$>$#1$<$/th$>$$<$/tr$>$\string\odskoc}%
  }

\def\vedecke{\zasah{Vědecké a odborné články}} 
\def\prehledove{\zasah{Přehledové články}} 
\def\odborne{\zasah{Odborné články}} 
\def\jine{\zasah{Jiné statě}} 
% Vědecká, přehledová, jiná stať
\def\zpravy{\zasah{Zprávy a informace}}
\def\pozvanky{\zasah{Pozvánky na akce}}
\addtolength{\voffset}{4.5mm}
%\def\zvyrazni#1{\textit{#1}}

\makeatletter
\renewcommand\@seccntformat[1]{\@nameuse{the#1}.\hspace{.5em}}
\makeatother

\def\bs{$\backslash$}
\usepackage{csts}
\widowpenalty=10000
\clubpenalty =10000

\ifnum\zahlavi=0
\overfullrule=0cm % 0cm
\else
\overfullrule=3cm
\fi

\def\z{\discretionary{-}{-}{-}}
\def\odskoc{\string\odskoc\space }
 %  impli-ca-tio-nal

%\DeclareRobustCommand\CS[1]
% {\def\next{#1}\leavevmode{$\cal C$}\kern-.3em
%  \lower .67ex\hbox{$\cal S$}\if E\next\kern-.045em\else\kern-.145em\fi#1}
%\newcommand\zapomen[1]{}
\usepackage{caption}
\captionsetup{justification=centerlast}% bylo justified, centering
%\captionsetup{tableposition=above}
%\captionsetup[table]{position=top}
%\captionsetup{type=table,position=above,skip=3pt}
%\usepackage{enumerate}
%\usepackage{bbm}


% Bohdalová
\newtheorem{defn}{Definition}[section]
\newtheorem{example}{Example}[section]

% Klicnarová
\newtheorem{thm}{Theorem}[section]
\newtheorem{theorem}{Theorem}[section]

% Friesl
%\usepackage{enumerate}
%\usepackage{amsmath}
%\newtheorem{theorem}{Tvrzení}[section]
%\newtheorem{theoremSK}{Tvrdenie}[section]
\newtheorem{defi}{Definition}[section]
\newtheorem{definition}{Definícia}[section]

%\theoremstyle{remark}
%\newtheorem*{remark}{Poznámka}
%\newtheorem*{remarkEN}{Remark}

% Dodany styl k cislu sekce pridava tecku, takze uprostred cisla
% tvrzeni pak jsou dve tecky za sebou... predefinuju si to:
\makeatletter
\renewcommand\thetheorem{\@arabic\c@section.\@arabic\c@theorem}
\renewcommand\thethm{\@arabic\c@section.\@arabic\c@thm}
\renewcommand\thedefi{\@arabic\c@section.\@arabic\c@defi}
\renewcommand\thedefn{\@arabic\c@section.\@arabic\c@defn}
\renewcommand\theexample{\@arabic\c@section.\@arabic\c@example}
\renewcommand\thedefinition{\@arabic\c@section.\@arabic\c@definition}
%\renewcommand\thetheoremSK{\@arabic\c@section.\@arabic\c@theoremSK}
\makeatother
%
% ******************  Matematické definice  *******************
\DeclareMathOperator{\e}{e}
\newcommand{\RR}{\mathbb R}
\newcommand{\ZZ}{\mathbb Z}
\newcommand{\NN}{\mathbb N}
\newcommand{\TT}{\mathbb T}
\DeclareMathSymbol{\gi}{\mathbin}{symbols}{"6A}
\newcommand{\defs}{\mathrel{:=}}
\DeclareMathOperator{\E}{E}
\DeclareMathOperator{\PP}{P}
\newcommand{\tu}{\widetilde u}
\newcommand{\ta}{\widetilde a}
\newcommand{\tb}{\widetilde b}
\newcommand{\tc}{\widetilde c}
\newcommand{\tX}{\widetilde X}
\def\dd{\,{\rm d}}
\def\rmd{{\rm d}}
\def\argmin{\operatorname{arg\,min}\,}
\def\argmax{\operatorname{arg\,max}\,}

% Zvára
%\usepackage{natbib}
%\usepackage{Sweave-mal}
%\newcommand{\FKvantil}[3]{\mbox{$F_{#1,#2}(#3)$}}



%\def\der{\mathrm{d}\;\!}
\usepackage{url}
\renewcommand{\UrlBreaks}{\do\-\do\.\do\_\do\/}

\newcommand{\mylink}[1]{\footnote{\texttt{<#1>}}}
\long\def\Abstract#1{%
  {\selectlanguage{english}\frenchspacing\medskip\noindent{\textbf{Abstract:}} #1}\par
  }

\long\def\Abstrakt#1{%
  {\selectlanguage{czech}\medskip\noindent{\textbf{Abstrakt:}} #1}\par
  }

\long\def\AbstraktSK#1{%
  {\selectlanguage{slovak}\medskip\noindent{\textbf{Abstrakt:}} #1}\par
  }

\def\upravalit{%
  \ifx\mujjazyk\macestina \def\refname{Reference}\fi%
  \ifx\mujjazyk\maslovenstina \def\refname{Literatúra}\fi%
  \ifx\mujjazyk\maanglictina \def\refname{References}\fi%
  }

\def\KeyWords#1{\par\noindent{\selectlanguage{english}\frenchspacing\textbf{Keywords:} #1.}\par\upravalit}
\def\KlicovaSlova#1{\par\noindent{\selectlanguage{czech}\textbf{Klíčová slova:} #1.}\par\upravalit}
\def\KlucovaSlova#1{\par\noindent{\selectlanguage{slovak}\textbf{Kľúčová slová:} #1.}\par\upravalit}

\partopsep=-4pt plus 1pt minus 1pt
\ifx
\def\cl{
\itemsep=2.5pt
\parskip=-1.5pt
%\partopsep=-6pt
%\topsep=-6pt
%\parsep=-60pt
}
\fi
\def\engl{\selectlanguage{english}
   \frenchspacing}
\def\ee{{\rm e}}   
\def\LGD{\mathop{\rm  LGD}}
\def\iid{{\rm i.i.d.}}
\def\var{\mathop{\rm var}}
\def\cov{\mathop{\rm cov}}
\def\RVC{\mathop{\rm RVC}\nolimits}
\def\MDA{\mathop{\rm MDA}\nolimits}
\def\Poi{\mathop{\rm Poi}\nolimits}
\def\PML{\mathop{\rm PML}\nolimits}


%\def\mleft{\textrm{\textlangle\;\!}}
%\def\mright{\textrm{\;\!\textrangle}}
\usepackage[normalem]{ulem}
%\ULdepth=1.5pt
%\let\href=\url

\ifx
\def\hvezdicka{*}
\def\citet#1#2{%
  \def\parameter{#1}%
  \ifx\hvezdicka\parameter
    \cite{#2}%
  \else
    \cite{#1}#2%
  \fi}
\fi



%\usepackage{eurosym}
%\let\texteuro=\officialeuro

% Skopal
%\definecolor{gray}{rgb}{0.4,0.4,0.4}
%\definecolor{darkblue}{rgb}{0.0,0.0,0.6}
%\definecolor{cyan}{rgb}{0.0,0.6,0.6}
\usepackage{listings}
%\ifx\relax

\lstset{
    %inputencoding=utf8,
    extendedchars=true,
    literate=%
    {á}{{\'a}}1
    {č}{{\v{c}}}1
    {ď}{{\v{d}}}1
    {é}{{\'e}}1
    {ě}{{\v{e}}}1
    {í}{{\'i}}1
    {ň}{{\v{n}}}1
    {ó}{{\'o}}1
    {ř}{{\v{r}}}1
    {š}{{\v{s}}}1
    {ť}{{\v{t}}}1
    {ú}{{\'u}}1
    {ů}{{\r{u}}}1
    {ý}{{\'y}}1
    {ž}{{\v{z}}}1
    {Á}{{\'A}}1
    {Č}{{\v{C}}}1
    {Ď}{{\v{D}}}1
    {É}{{\'E}}1
    {Ě}{{\v{E}}}1
    {Í}{{\'I}}1
    {Ň}{{\v{N}}}1
    {Ó}{{\'O}}1
    {Ř}{{\v{R}}}1
    {Š}{{\v{S}}}1
    {Ť}{{\v{T}}}1
    {Ú}{{\'U}}1
    {Ů}{{\r{U}}}1
    {Ý}{{\'Y}}1
    {Ž}{{\v{Z}}}1
}

\lstset{
  %extendedchars=true,
  %inputencoding=utf8,
  breakatwhitespace=true,
  breaklines=true, 
  basicstyle=\ttfamily\footnotesize,
  columns=fullflexible,
  showstringspaces=false,
  commentstyle=\color{gray}\upshape,
  keepspaces=true
}

\lstdefinelanguage{XML}
{
  morestring=[b]",
  morestring=[s]{>}{<},
  morecomment=[s]{<?}{?>},
  stringstyle=\color{black},
  identifierstyle=\color{darkblue},
  keywordstyle=\color{cyan},
  morekeywords={xmlns,version,type}% list your attributes here
}
%\fi

% Hasilova
%\usepackage{subfigure}
%\newtheorem{priklad}{Příklad}
%\newtheorem{res}{Řešení}
\usepackage{bm}
%\usepackage{fancyvrb}

\newdimen\zde
\zde=\textwidth
\advance\zde by -2\fboxrule

\selectlanguage{czech}
\hyphenation{MOVISS ROBUST}

\selectlanguage{english}
\hyphenation{}

\def\malE{\mathsf{E}\;\!}
%\def\velkeE{\mathsf{E}}

% 2020-03-06, malipivo, tři řádky
\renewcommand\rmdefault{cmr}
\renewcommand\sfdefault{cmss}
\renewcommand\ttdefault{cmtt}


\begin{document}
\selectlanguage{czech}
\hyphenation{log-nor-mál-ní log-nor-mál-ním Bes-ky-dech dal-ších}
\shorthandoff{-}
%\lefthyphenmin=2
%\righthyphenmin=2

\def\mujclear{
  \setcounter{part}{0}
  %\setcounter{chapter}{0}
  \setcounter{section}{0}
  \setcounter{subsection}{0}
  \setcounter{subsubsection}{0}
  \setcounter{paragraph}{0}
  \setcounter{subparagraph}{0}
  %\setcounter{page}{0}
  \setcounter{equation}{0}
  \setcounter{figure}{0}
  \setcounter{table}{0}
  \setcounter{footnote}{0}
  \setcounter{mpfootnote}{0}
  %\setcounter{enumi}{0}
  %\setcounter{enumii}{0}
  %\setcounter{enumiii}{0}
  %\setcounter{enumiv}{0}
  }

\def\vlozstranu{\clearpage}
\def\macestina{czech}
\def\maslovenstina{slovak}
\def\maanglictina{english}

\def\vloz#1#2{%
  \vlozstranu
  \mujclear
\def\ulozsi{#1}%
\begingroup
  \label{#1}%
  \selectlanguage{#2}%
  \frenchspacing % i pro angličtinu, jak je to nezvyk pro našinec
  \def\mujjazyk{#2}%
%  \iflanguage{czech}{\def\refname{Literatura}}{}%
%\ifnum\language=16\gdef\refname{Literatura}\fi%
  \input{zdrojaky/#1}\par\relax%
\endgroup
  %\par\mujnazevEN
  \gdef\vlozstranu{\clearpage}
  %\closeme % !experiment
  %\newpage% !experiment
  }

\def\extras{%
% začátek experimentu
%\closeout\maosnova
%\closeout\maosnovab
%\newwrite\maosnova
%\openout\maosnova=maosnova.tex
%\newwrite\maosnovab
%\openout\maosnovab=maosnovab.tex
% konec experimentu
  \write\maosnova{\emph{\mujauthor}\string\odskoc\space\mujnazevCS\string\pr{\ulozsi}}%
  \write\maosnovab{$<$tr$>$\string\odskoc$<$th$>$\mujauthor$<$/th$>$\string\odskoc\space$<$td$>$\mujnazevPR$<$/td$>$\string\odskoc\space$<$td$>$\mujnazevDR$<$/td$>$\string\pr{\ulozsi}}%
  }

\ifx
\def\notes#1#2#3{%
\begingroup
\def\thefootnote{}
\footnotetext{\kern-1.8em Doručeno redakci: #1, imprimatur: #2.\\
MSC2010: #3, DOI: \textsf{10.5300/IB/2011-2/\thepage}}%
\endgroup
}
\def\medoi#1{\textsf{doi:\,#1}}
\fi

\newcommand{\nazevEN}[1]{
    \noindent
    \raggedright
    \large\textbf{\MakeUppercase{#1}}
    \medskip
    }
\newcommand{\nazevCS}[1]{
    \noindent
    \raggedright
    \large\textbf{\MakeUppercase{#1}}
    \medskip
    }

\makeatletter
\def\dotpfill{%
  \leavevmode%
  \leaders\hb@xt@.33em{\hss.\hss}\hfill%
  \kern\z@}
\makeatother 
\newcommand{\pr}[1]{\,\dotpfill\makebox[4mm][r]{\pageref{#1}}\\[6pt]}

\newwrite\maosnova
\openout\maosnova=maosnova.tex

\newwrite\maosnovab
\openout\maosnovab=maosnovab.tex
\write\maosnovab{$<$table border=$"$1$"$$>$\string\odskoc\string\prostor\string\odskoc}

\def\mastrana{%
  \def\vlozstranu{}%\vfill
  }

\def\closeme{%
\newwrite\stran
\immediate\openout\stran=stran.tex
\immediate\write\stran{\stran=\thepage}
\closeout\stran
%\write\maosnovab{\string\odskoc$<$/table$>$\string\odskoc\string\prostor}
}


\ifx\relax
\IfFileExists{maosnova2.tex}
  {{\parindent=0pt\textbf{Pracovní osnova\ldots}\medskip\par\input{maosnova2.tex}}}
  {Zde bude obsah\ldots}
\fi


\newcount\cislo
\newcount\rocnik
\IfFileExists{cislo.tex}
  {\input cislo.tex%\zahlavi=1
  }{%\zahlavi=0
  \cislo=1 % 1 nebo 3
  }
\cislo=1

%\pgfmathparse{int(\the\rocnik+2008-2000)}
%\let\mezivypa=\pgfmathparse
%\pgfmathparse{int(\the\rocnik+2008)}
%\let\mezivypb=\pgfmathparse
\newcount\hodnotaA
\newcount\hodnotaB

\write\maosnovab{$<$tr$>$\string\odskoc$<$th colspan=$"$4$"$$>$\string\odskoc}
\write\maosnovab{$<$a name=$"$\string\the\hodnotaA-\string\the\cislo$"$$>$\string\the\cislo/\string\the\hodnotaB$<$/a$>$\string\odskoc}
\write\maosnovab{$<$a title=$"$http://statspol.cz/bulletiny/ib-\string\the\hodnotaB-\string\the\cislo-web.pdf$"$ href=$"$ib-\string\the\hodnotaB-\string\the\cislo-web.pdf$"$$>$Bulletin v pdf$<$/a$>$\string\odskoc$<$/tr$>$\string\odskoc}

\renewcommand\arraystretch{1.25}
%\allowdisplaybreaks
\fboxsep=0pt
\def\logit{\mathop{\rm logit}}
\def\malee{{\rm e}}

%\def\mujnazevCS{}
%\def\mujnazevEN{}
%\def\mujnazevPR{}
%\def\mujnazevDR{}
%\def\mujauthor{}
\graphicspath{
   %{zdrojaky/vozar/}
   %{zdrojaky/cess2020/}
   %{zdrojaky/rytir/obr/}
   %{prichozi/striz/}
	 {zdrojaky/mosna/obr/}
	 {zdrojaky/knihovnicka/obr/}
	 {zdrojaky/andel/obr/}
	 {zdrojaky/dohnal/obr/}
	 {zdrojaky/mapa/obr/}	 
   {zdrojaky/strizkorona/obr/}
   }
\def\TeX{T\kern-.1667em\lower.5ex\hbox{E}\kern-.125emX} % bylo na \protect\TeX ???

\def\xbonus{}
\ifnum\cislo=1
\zpravy
\vloz{vybor/vybor}{czech}
\vloz{vencalek/vencalek}{czech}
\vedecke
%\pozvanky
%\vloz{mosna/mosna-OPRAVENE1}{czech}
\vloz{mapa/mapa}{czech}
\zpravy
\vloz{dohnal/dohnal}{czech}
\vloz{knihovnicka/knihovnicka}{czech}
\vloz{strizkorona/strizkorona}{czech}
\vloz{andel/andel}{czech}
  \fi
\closeme

\write\maosnovab{\string\odskoc$<$/table$>$\string\odskoc\string\prostor} %\newpage
\closeout\maosnova
\closeout\maosnovab %\closeme
\write18{cp -f maosnova.tex maosnova2.tex} %\newpage

\end{document}



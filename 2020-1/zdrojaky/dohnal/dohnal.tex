% !TEX TS-program = LuaLaTeX
% !TEX encoding = UTF-8 Unicode 
% !TEX root = ../../!hlavni-soubor.tex

\gdef\mujnazevCS{Mikuklášské setkání členů České statistické společnosti}
\nazev{Mikuklášské setkání členů\\[2pt] České statistické společnosti}

\gdef\mujnazevEN{Meeting of the Czech Statistical Society on Saint Nicholas Day}
\nazev{Meeting of the Czech Statistical\\[2pt] Society on Saint Nicholas Day}

\gdef\mujnazevPR{\mujnazevCS}
\gdef\mujnazevDR{\mujnazevEN}

\gdef\mujauthor{Gejza Dohnal}
\author{\mujauthor}

\Email{gejza.dohnal@csc-sro.cz}

\medskip

\noindent
Svátek svatého Mikuláše už několik let není pouze svátkem dětí, těšících se na mikulášskou nadílku a~zároveň se obávajících umouněného čerta, před kterým budou muset odvyprávět básničku, zazpívat písničku či zatančit taneček. I~když z~dávné doby, kdy jsem ještě s~kamarády chodil po městě převlečený za jednoho z~té mikulášské trojice (zpravidla čerta) a~také z~doby méně dávné, leč přesto již dávné, kdy jsem chodil po městě se svými dětmi a~potkávali jsme mikulášsko-andělsko-čertovské trojice, jsem měl pocit, že děti se nejvíc obávaly důstojnější a~z~jejich pohledu přísné postavy svatého Mikuláše. No zřejmě máme blíž k~peklu než do nebe. 

Tedy tento svátek je už několik let (ani už nevím kolik) také svátkem českých statistiků. V~tento den, nebo těsně v~jeho okolí, se totiž -- co moje paměť sahá -- pravidelně pořádá Mikuklášské setkání členů naší České statistické společnosti. Název \uv{Mikuklášské} už není překlepem. Byl to překlep na samém počátku této tradice ale zalíbil se nám a~protože v~prvních letech na toto setkání přicházel s~velkým humbukem i~můj syn, zakuklený v~masce čerta, tento název zůstal a~stal se oficiálním. 

V~minulých letech Mikuklášské setkávání probíhalo v~respiriu Matema\-ticko\z fyzikální fakulty Karlovy Univerzity v~Karlíně. V~posledních letech se jej účastnilo pouze pár \uv{skalních} za občasných návštěv členů katedry pravděpodobnosti a~matematické statistiky sídlící v~této budově. O~to větší bylo moje překvapení, když jsem přišel minulý pátek do zasedací místnosti Fakulty informatiky a~statistiky VŠE a~jen stěží jsem našel volné místo. Ukázala se další výhoda mít -- z~pragocentrického hlediska -- přespolního předsedu: přijelo s~ním i~několik mimopražských kolegů. 

Já se na setkání dostavil s~více než hodinovým zpožděním díky plánované lékařské prohlídce. Nicméně jsem měl výhodu v~tom, že jako jediný účastník jsem mohl doslova sledovat příspěvek Zdeňka Fabiána (ÚI AV ČR) \em Opožděná reportáž\em\ v~přímém přenosu. Volal mi totiž už do ordinace mého ošetřujícího lékaře s~naléhavou prosbou, kde že se to vlastně koná? Milý Zdeněk totiž~-- jako každoročně -- vyrazil suverénně do respiria v~karlínské budově MFF. Když zjistil, že tam nikdo není, začal pátrat a~zjistil, že musí do areálu Vysoké školy ekonomické na Žižkov. Ale tam narazil podruhé. Ztratil se totiž v~labyrintu budov VŠE\footnote{Na jeho obranu je však nutno dodat, že ztratit se v~labyrintu VŠE pro člověka zvenku není nic neobvyklého a~stává se to i~nám, kteří tyto budovy navštěvujeme častěji než Zdeněk.}
%Já se na setkání dostavil s~více než hodinovým zpožděním díky plánované lékařské prohlídce. Nicméně jsem měl výhodu v~tom, že jako jediný účastník jsem mohl doslova sledovat příspěvek Zdeňka Fabiána (ÚI AV ČR) \em Opožděná reportáž\em\ v~přímém přenosu. Volal mi totiž už do ordinace mého ošetřujícího lékaře s~naléhavou prosbou, kde že se to vlastně koná? Ztratil se v~labyrintu budov Vysoké školy ekonomické v~Praze\footnote{Na jeho obranu je však nutno dodat, že ztratit se v~labyrintu VŠE pro člověka zvenku není nic neobvyklého a~stává se to i~nám, kteří tyto budovy navštěvujeme častěji než Zdeněk.} 
i~se svým malým psíkem, kterého vzal s~sebou. Část času strávil diskuzí s~kýmsi o~tom, která budova je ta nová a~která je stará. Volal ještě několikrát, než se mi podařilo jej nasměrovat do nové budovy VŠE, 4. patro doprava a~ještě jednou doprava do místnosti~-- 467 nebo 473? No, a~to už našel. Svůj čas k~přednášce ovšem tímto vyčerpal, a~tak jsem jeho opožděnou reportáž slyšel pouze já v~telefonu. Díky pozdnímu příchodu jsem přišel o~první tři určitě nejzajímavější přednášky z~oblasti analýzy prostorových dat kolegů Jaromíra Antocha (MFF UK Praha), Tomáše Formánka (FIS VŠE Praha) a~Patrície Martinkové (ÚI AV ČR, PedF UK Praha). 

Zatímco první z~nich představil matematický model pro%
storových dat
a~ukázal možnosti testování změn v~tomto modelu, konkrétně možnosti \em Detekce změn v~kapitálové struktuře\em, další dva ve svých příspěvcích ukázali praktické použití na reálných datech. Kolega Formánek ve svém příspěvku \em Využití postupu prostorové ekonometrie při analýze dopadu zavádění obnovitelných zdrojů na růst HDP\em\ ukazoval zpracování prostorových dat o~znečištění v~11 evropských zemích a~kolegyně Martinková ukázala \em Pokroky v~analýze odlišného fungování položek znalostních testů\em\ v~rámci zjišťování změn v~kompetencích k~učení u~žáků základních škol a~víceletých gymnázií. Nicméně nakonec jsme já, Zdeněk a~jeho psík dorazili na místo zrovna v~době první přestávky a~nelitovali jsme.
Dostali jsme kávu, žižkovský štrůdl a~od předsedy společnosti mikuklášskou nadílku.

Po přestávce nám Ondřej Vencálek (PřF UP Olomouc) s~laskavým před\-ne\-sem ukazoval, jak lze zkomplikovat jednu z~nejstarších známých úloh o~roz\-dě\-le\-ní sázky. Ve svém příspěvku \em Možná přijde i~Bayes\em\ nám ukázal, jak lze celkem přehledné a~jednoduché řešení libovolně \uv{znepřehlednit} použitím Bayesovského přístupu. Myslím, že řadu posluchačů pobavil.

V~dalším příspěvku Marek Malý (SZÚ Praha) uvedl své \em Poznámky k~nesprávnému použití statistiky\em\ a~o~častých chybách při interpretaci statistických výsledků. To je samozřejmě \uv{stará písnička}, o~které s~oblibou mluvili kdysi kolegové Žváček, Tvrdík a~Kupka v~souvislosti s~prezentací výsledků statistických šetření v~médiích. Marek Malý se zaměřil na medicínská data, s~jejichž interprety v~lékařských kruzích má bohaté zkušenosti.

Z~trošku jiného soudku a~s~trochu vážnějším naladěním jsme si vyslechli \em Jak se stanovují platby na podporu zemědělských podniků hodpodařících v~mé\-ně příznivých oblastech ČR\em.  Tuto pohádku o~tom, jak našim zemědělcům spravedlivě rozdělovat zasloužené dotace přednesl Tomáš Hlavsa (PEF ČZU Praha) a~dozvěděli jsme se přitom, ve kterých částech naší republiky jsou nejchudší zemědělci a~podle jakého klíče jsou jim dotace kráceny. Čím větší rozlohu obhospodařuješ, tím více ti zkrátíme dotační titul. Určitě je to jedna z~cest, jak ze všech špatných modelů vybrat ten lepší.\footnote{\ldots{}all models are approximations. Essentially, all models are wrong, but some are useful. However, the approximate nature of the model must always be borne in mind\ldots{} (George E.\,P.\,Box)}
 
Příspěvky Karla Zváry (PřF UK Praha) a~Daniela Hlubinky (MFF UK Praha) byly věnovány postřehům z~výuky. Jako témata posloužily Karlovy \em Poznámky STAršího KANtora\em\ k~vysvětlování principu testování hypotéz a~Danielovy poznámky ke \em Statistickému odhadu ve výuce\em. Nicméně obecné závěry platí o~jakémkoli tématu při výuce matematické statistiky a~prav\-dě\-po\-dob\-nos\-ti (nejen) pro nestatistiky. Kolegové prezentovali zjištění, že už ani nemá cenu požadovat po studentech znalosti vzorců a~formulí (řadu z~nich už jim ani neukazují), ale pochopení základních principů. To je ovšem na celé výuce to nejtěžší -- vysvětlit základní pojmy a~principy tak, aby je studenti pochopili. Padly zde odkazy na mistry výuky Jiřího Anděla a~Josefa Machka. Nicméně i~Karel a~Daniel jsou mistři svého řemesla a~jejich zkušenosti a~postřehy jsou pro nás velmi cenné.   

Program byl letos pouze odpolední, ale zato delší. Po odborném programu Zdeněk Fabián stejně jako v~letech minulých vytáhl kytaru a~všichni zbylí kolegové (a~někteří i~nově příchozí právě na tuto část programu) s~napětím očekávali nové texty jeho písniček reagující na aktuální dění. Pravda, bylo jich (těch nových textů či aktualizace starších) trochu méně než v~minulosti, ale byly. No a~tak začala druhá část Mikukláše. Ač jsem doma manželce sliboval, že letos přijdu brzy, ještě před večeří, nakonec jsem byl rád, že jsem došel domů ještě v~ten den před půlnocí. 

Tak takhle jsem viděl letošní Mikuklášské setkání 2019.
\vfil

\newdimen\maltemp \maltemp=\textwidth
\advance\maltemp by -2pt
\fboxrule=1pt
\fboxsep=0pt

\noindent
\fbox{\includegraphics[trim=0pt 25mm 0pt 4cm, clip, width=\maltemp]{DSCN2752.JPG}}

\newpage
\noindent
\fbox{\includegraphics[trim=0pt 0pt 0pt 2cm, clip, width=\maltemp]{DSCN2745.JPG}}%
\medskip

\noindent
\fbox{\includegraphics[width=\maltemp]{DSCN2747.JPG}}


% !TEX TS-program = LuaLaTeX
% !TEX encoding = UTF-8 Unicode 
% !TEX root = ../../!hlavni-soubor.tex

\gdef\mujnazevCS{Postřehy kolem koronaviru}
\nazev{\mujnazevCS}

\gdef\mujnazevEN{Notes on coronavirus}
\nazev{\mujnazevEN}

\gdef\mujnazevPR{\mujnazevCS}
\gdef\mujnazevDR{\mujnazevEN}

\gdef\mujauthor{Pavel Stříž}
\author{\mujauthor}

%\Adresa{}

\Email{pavel@striz.cz}

%\Abstrakt{}
%\smallskip
%\KlicovaSlova{}

%\Abstract{}
%\smallskip
%\KeyWords{}

\medskip

\noindent
Od listopadu 2019 Čínu, od ledna 2020 svět, papírově od 1. března 2020 trápí i~Českou republiku koronavirus alias COVID-19 alias SARS-CoV-2. Byl odložen ROBUST 2020 (\url{https://www.statspol.cz/robust-2020/}). % a svět bojuje. 

Pro badatele však může být zajímavé, že existují otevřené datové sady (\url{https://onemocneni-aktualne.mzcr.cz/covid-19}) a poměrně podrobné informace o dění ve světě (\url{https://opendatawatch.com/what-is-being-said/data-in-the-time-of-covid-19/}).
%
Mou pozornost zaujalo video o~simulacích (\url{https://www.youtube.com/watch?v=gxAaO2rsdIs}) a 2 %dva výpo\-četní 
projekty zmíněné v článku (\url{https://www.lupa.cz/clanky/pomohou-hraci-online-}\\ \texttt{hry-nebo-distribuovane-vypocty-porazit-koronavirus/}). %Zkusil jsem si nainstalovat a vyzkoušet.

\enlargethispage{\baselineskip}


\subsection*{Folding@Home}

Prvním z projektů je Folding@Home (\url{https://foldingathome.org/}), vytvořený profesorem %Vijay 
Pandem pod záštitou Stanfordské univerty, projekt je nyní pod patronátem Washingtonské univerzity.  

Na Xubuntu mi po instalaci tří souborů vznikly ikony FAHControl a~FAHViewer. V jádru se jedná o zapůjčení výpočetní síly a metodu hrubé síly (brute force method) v~pozadí. Rozhraní vypadá takto.

\fboxsep=0pt
\fboxrule=1pt
\newdimen\delka
\delka=\textwidth
\advance\delka by -2\fboxrule

\noindent
\hfil
\fbox{\includegraphics[width=\delka]{fahcontrol.png}}

%Při instalaci je dobré předvolené možnosti ponechat, pokud se odškrtne spuštění 

\subsection*{Rosetta@Home}

Podobný projekt je Rosetta@Home založený na programu BOINC (\url{https://boinc.berkeley.edu}), kde si uživatel volí projekt, který se bude počítat.



%\newpage
\subsection*{Foldit}

Možná klíčový projekt je Foldit (\url{https://fold.it/}), který bioinformatické problémy transformuje na hry. Zakladatelem je Seth Cooper.

Instalace byla bez problémů, pro hackery her dokonce adresáře plné zajímavých souborů (PDB, TXT ad.). \TeX isty možná zaujme možnost užití Lua, viz na fóru recepty/recipes (\url{https://fold.it/portal/recipes}).

Petr Hudeček, spolužák ze Zlína, se do hraní intenzívně pustil. Zmínil, že mu počítač pomáhá v závěrečné fázi složení proteinů, předtím však musí zasahovat on. Zde je náhled na rozhraní pokročilého kola. Trochu mi to připomnělo začátek filmu Muž bez stínu (Hollow Man, 2000).%
\smallskip

\noindent
\fbox{\includegraphics[width=\delka]{foldit2.png}}

\enlargethispage{\baselineskip}



\subsection*{Detekce viru}

Problém je i detekce koronaviru z rentgenových snímků, více o tom na stránkách (\url{https://www.pyimagesearch.com/2020/03/16/detecting-covid-}\\ \texttt{19-in-x-ray-images-with-keras-tensorflow-and-deep-learning/}). 

OpenCV a práci Adriana Rosebrocka (PyImageSearch) asi netřeba blíž představovat, znají ji především uživatelé mikropočítačů typu Raspberry Pi.



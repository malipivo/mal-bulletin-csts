% !TEX TS-program = LuaLaTeX
% !TEX encoding = UTF-8 Unicode 
% !TEX root = ../../!hlavni-soubor.tex

\gdef\mujnazevCS{Zpráva o činnosti České statistické společnosti v roce 2019}
\nazev{Zpráva o činnosti České statistické\\[2pt] společnosti v roce 2019}

\gdef\mujnazevEN{Czech Statistical Society in 2019}
\nazev{\mujnazevEN}

\gdef\mujnazevPR{\mujnazevCS}
\gdef\mujnazevDR{\mujnazevEN}

\gdef\mujauthor{Ondřej Vencálek}
\author{\mujauthor}

\Email{ondrej.vencalek@upol.cz}



\section*{Členská základna}
K 31.\,12.\,2019 měla ČStS 203 členů. V průběhu roku 2019 přibylo 19 nových členů, naopak ve 12 případech bylo členství z různých důvodů ukončeno. Počet členů se tedy v průběhu roku zvětšil o 7. Zejména vysoký počet nových členů je dobrou zprávou pro naši společnost.   

\section*{Akce pořádané či spolupořádané ČStS}

\begin{itemize}
\itemsep=0pt
\item[1)] \textbf{Členská schůze ČStS} se uskutečnila dne 31. ledna 2019 v prostorách ČSÚ v Praze. Zúčastnilo se 25 členů. Po přečtení zpráv o činnosti a o hospodaření se uskutečnily volby předsedy, výboru a revizora společnosti. Následovala přednáška doc. RNDr. Jany Strakové, Ph.D., z Pedagogické fakulty Univerzity Karlovy v Praze na téma \textit{Mezinárodní výzkumy výsledků vzdělávání.} 
\item[2)] \textbf{Odborný seminář ČSÚ a ČStS Století statistiky} se uskutečnil 27. února 2019 v prostorách ČSÚ v Praze v rámci oslav sta let nezávislé státní statistické služby na území současné České republiky. Účastníci si vyslechli 5 odborných příspěvků.
\item[3)] \textbf{Workshop Modelling Smart Grids 2019} se uskutečnil ve dnech 19. až 21. září 2019 v prostorách TU v Liberci a ČVUT v Praze. Celkem 37 účastníků ze 6 zemí mělo možnost vyslechnout 21 odborných příspěvků. Šlo o pátý ročník mezinárodního workshopu, který se koná každoročně od roku 2015 v rámci projektu Strategie AV21 České akademie věd a ve spolupráci s ČStS, Univerzitou Karlovou a s Českým vysokým učením technickým. V roce 2019 se na organizaci podílela i Technická univerzita v Liberci. 
\item[ 4)] \textbf{Konference STAKAN 2019 (Statističtí kantoři)} zaměřená na výuku statistiky na vysokých i středních školách se uskutečnila ve dnech 11. až 13. října 2019 ve Křtinách. Spoluorganizátorem akce byla Slovenská štatistická a demografická spoločnosť. Celkem 48 účastníků mělo možnost vyslechnout 20 odborných příspěvků.
\item[    5)] \textbf{Mikuklášský statistický den} se uskutečnil dne 6. prosince 2019 v prostorách VŠE v Praze. Rekordních 35 účastníků mělo možnost vyslechnout 8 odborných příspěvků.   
    \end{itemize}
    
\section*{Spolupráce s Českým statistickým úřadem}
Český statistický úřad je dlouhodobě partnerem ČStS a podporuje její činnost například zajištěním tisku Informačního bulletinu ČStS, poskytnutím sídla společnosti a prostoru pro konání členských schůzí ČStS. V průběhu roku 2019 ČSÚ připravil několik akcí k výročí 100 let moderní státní statistické služby, včetně výše uvedeného únorového odborného semináře Století statistiky a slavnostní recepce, která se pod záštitou prezidenta republiky konala za účasti představitelů vlády ČR a vedení Eurostatu dne 19. září 2019 v Praze. Obou těchto akcí se zúčastnili zástupci ČStS. 

\section*{Členství v mezinárodních organizacích}
Naše společnost je od roku 2011 členem federace evropských národních statistických společností FENStatS (The Federation of European National Statistical Societies). V průběhu roku 2019 FENStatS schválil vznik tzv. akreditací evropských statistiků. O zavedení tohoto systému v rámci ČR bude jednat výbor ČStS. Pro naši společnost je rovněž důležité zavedení členského příspěvku, který od roku 2019 jednotlivé národní společnosti sdružené ve FENStatS platí. Na námi placený příspěvek ve výši 200 eur jsme úspěšně čerpali dotaci od Akademie věd ČR. Podobně hodláme postupovat i v tomto roce. Valné shromáždění FENStatS se konalo 22. srpna 2019 v Kuala Lumpur v Malajsii. ČStS zastupoval předseda společnosti, který stejně jako někteří zástupci dalších národních společností využil možnosti účasti prostřednictvím Skype.

Naše společnost je rovněž členem sdružení národních společností středoevropského regionu V7. Setkání představitelů těchto společností se konalo 27. září 2019 v Lublani. ČStS zastupoval její předseda. Více informací je k nalezení v Informačním bulletinu 3/2019.

\section*{Členství v Radě vědeckých společností ČR}
ČStS je členem Rady vědeckých společností ČR (RVS). Na plenárním zasedání, které se uskutečnilo 24. dubna 2019 v Praze, reprezentoval naši společnost místopředseda společnosti Marek Malý. V roce 2019 jsme, jak již bylo uvedeno výše, prostřednictvím RVS poprvé úspěšně žádali o dotaci na členský příspěvek ve sdružení FENStatS.  

\section*{Další činnost}
\begin{itemize}
\itemsep=0pt
\item Byla vydána 4 čísla Informačního bulletinu ČStS.
\item Byly rozesílány informační emaily členům společnosti.
\item Byla připravena nová webová stránka společnosti. ČStS se stala formálním vlastníkem domény \url{statspol.cz}, kterou až dosud vlastnil Jiří Žváček, resp. jeho dědic. Na přelomu roků 2019/2020 byla podepsána dohoda o spolupráci mezi ČStS a VŠB-TU Ostrava, která bude zajišťovat technický provoz domény \url{statspol.cz}, kterou až dosud bezplatně zajišťovala společnost \url{Nipax.cz}.
\item Byly připravovány Statistické dny v Brně, akce se však nakonec neuskutečnila.
\item Proběhlo setkání zástupců ČStS s delegací Regionálního statistického úřadu čínské provincie Če-ťiang. Viz zpráva v Informačním bulletinu 2/2019.
\item Začaly přípravy konference Robust 2020.
\item Proběhly přípravy mezinárodní konference Conference of European Statistics Stakeholders (CESS) 2020, která však byla jednostranně ze strany EUROSTATu zrušena.
\item Proběhla jednání s vedením ČSÚ o budoucí podobě oborových cen.
\end{itemize}

\section*{Akce chystané v roce 2020}
\begin{itemize}
\itemsep=0pt
\item 4.\,2.\,2020 – členská schůze v Praze.
\item 7. až 12.\,6.\,2020 -- dvacátá první letní škola JČMF ROBUST 2020 v Bardějově na Slovensku.
\item Září 2020 -- usilujeme o zorganizování dalšího workshopu Modelling Smart Grids.
\item Kolem 6.\,12.\,2020 -- Mikuklášský den ČStS v Praze.
\end{itemize}
\bigskip

\noindent
V Olomouci dne 30. ledna 2020					

\hspace*{6cm} \hfil Ondřej Vencálek\par
\hspace*{6cm} \hfil předseda ČStS

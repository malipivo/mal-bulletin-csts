% !TEX TS-program = LuaLaTeX
% !TEX encoding = UTF-8 Unicode 
% !TEX root = ../../!hlavni-soubor.tex

\gdef\mujnazevCS{Členská schůze České statistické společnosti v~roce 2020}
\nazev{Členská schůze České statistické\\[2pt] společnosti v~roce 2020}

\gdef\mujnazevEN{ANNUAL MEMBERS' MEETING OF THE CZECH STATISTICAL SOCIETY IN 2020}
\nazev{ANNUAL MEMBERS' MEETING OF THE CZECH\\[2pt] STATISTICAL SOCIETY IN 2020}

\gdef\mujnazevPR{\mujnazevCS}
\gdef\mujnazevDR{\mujnazevEN}

\gdef\mujauthor{Ondřej Vencálek}
\author{\mujauthor}

%\Adresa{}

\Email{ondrej.vencalek@upol.cz}

\medskip

\noindent
Členská schůze České statistické společnosti se uskutečnila dne 4. února 2020 v prostorách Českého statistického úřadu v~Praze. Pětadvacet účastníků členské schůze se sešlo především, aby vyslechlo zprávy o~činnosti a~hospodaření společnosti v uplynulém roce a~také odbornou přednášku. 

Jako první promluvil k účastníkům schůze předseda Českého statistického úřadu Marek Rojíček. Ten v krátkosti představil novou podobu oborových cen, které ČSÚ dosud uděloval za mimořádného přínosu pro rozvoj statistiky. Nové pojetí oborových cen by mělo být širší. Vyhlášení prvních nositelů oborových cen je plánováno u~příležitosti Světového dne statistiky v říjnu tohoto roku.

Zprávu o~činnosti společnosti v~roce 2019, jejíž úplné znění zveřejňujeme v tomto čísle Informačního bulletinu, přečetl předseda společnosti Ondřej Vencálek. Zdůraznil zejména vysoký počet nových členů (devatenáct) a~velký zájem o~pořádané akce společnosti. Hospodář společnosti Tomáš Löster poté přečetl zprávu o~hospodaření ČStS v~roce 2019 (k~níž se kladně vyjádřila revizorka společnosti doc. Blatná) a~návrh rozpočtu na rok 2020. Všechny přednesené zprávy byly členskou schůzí schváleny.

Po krátké přestávce následovala odborná přednáška Jana Koláčka z Přírodovědecké fakulty Masarykovy Univerzity v~Brně na téma \textit{Vybrané aspekty analýzy funkcionálních dat} přehledně a~poutavě představující základy této moderní statistické metodologie. 



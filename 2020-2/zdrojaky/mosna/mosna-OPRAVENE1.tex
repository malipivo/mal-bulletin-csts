% !TEX TS-program = LuaLaTeX
% !TEX encoding = UTF-8 Unicode
% !TEX root = ../../!hlavni-soubor.tex

\newdimen\delkas \delkas=\textwidth
\fboxrule=1pt
\fboxsep=0pt
\advance\delkas by -2\fboxrule


\gdef\mujnazevCS{Neparametrické metody a~program Statistica}
\gdef\mujnazevEN{Nonparametric methods and Statistica software}
\gdef\mujnazevPR{\mujnazevCS}
\gdef\mujnazevDR{\mujnazevEN}
\gdef\mujauthor{František Mošna}

\bonushyper

\nazev{Neparametrické metody\\[2pt] a~program Statistica}

\nazev{Nonparametric methods and\\[2pt] Statistica software}

\author{\mujauthor}

\Adresa{Česká zemědělská univerzita v~Praze, Provozně ekonomická fakulta, Kamýcká 129, 165\,21 Praha~6}

\Email{mosna@pef.czu.cz}

\Abstrakt{Vhodným doplňkem výuky statistiky na vysokých školách bývá užití některých statistických programů.
Tyto nástroje poskytují mnohdy výsledky, které se ne vždy shodují se vzorci v učebnicích, což platí zejména u neparametrických metod.
Například u výsledků Wilcoxonova dvouvýběrového testu se uplatňují korekce na spojitost či na shody, přičemž některé učebnice je ani nezmiňují.
Také pro výpočet Spearmanova korelačního koeficientu lze užívat dvou vzorců a výsledky se mohou lišit.
Cílem příspěvku je ukázat u~těchto dvou neparametrických metod, které hodnoty zde počítá a~poskytuje program Statistica.}
\smallskip
\KlicovaSlova{neparametrické metody, Spearmanův korelační koeficient, Wilcoxonův dvouvýběrový test, program Statistica}

\Abstract{Use of some statistical program is a~suitable complement to the teaching of statistics at universities.
These tools often provide results which do not always coincide with formulas in textbooks, especially in the case of nonparametric methods.
For instance, the continuity correction and correction for ties are applied  at Wilcoxon rank-sum test, although they are not even
mentioned in some textbooks.
Also, two formulas can be used to calculate Spearman correlation coefficient and the results may be different.
The aim of this contribution is to show for these two nonparametric methods, which values are calculated and provided by Statistica program.}
\smallskip
\KeyWords{nonparametric methods, Spearman correlation coefficient, Wilcoxon rank-sum test, Statistica software}


\section{Výpočetní technika ve výuce statistiky}

Součástí kurzů statistiky na vysokých školách bývá zpravidla také seznámení s~nějakým statistickým programem nebo s~vhodnými statistickými prvky
výpočetní techniky. Mezi nejvíce užívané programy patří SPSS, Statistica, SAS, bezplatné programy R, JAMOVI, JASP a~podobně. Provádění  statistických
výpočtů umožňují také Microsoft Excel, Mathematica, Matlab a~další produkty.

Užívání statistických programů může být někdy spojeno s~jistým nebezpečím.
Uvedené programy nabízejí velké množství výpočtů a~grafů a~studenti se v~nich ne vždy dovedou orientovat. Někdy neumí vložit data, mají problémy se zadáváním správných požadavků,
se čtením a~získáváním poskytovaných informací.
 Mnohdy nerozumí, co vlastně obdrželi, nejsou schopni výsledky správně vyhodnotit a~následně interpretovat.

 Seznámení s~počítačovými programy je jistě vhodnou součástí výuky.
   Před tím je však potřeba správně porozumět principům statistických metod a~vědět, co vlastně tyto nástroje
provádějí, co počítají, co znamenají čísla, která programy poskytují.
Bývá vhodné prověřit si výpočty nejprve na jednoduchých datech velmi malého rozsahu a~porovnat čísla získaná \uv{ručně}
s~hodnotami získanými pomocí počítače. Tím docházíme k~porozumění výsledkům, které programy nabízejí. Následně lze užívat programy
pro větší \uv{skutečná} (z~praxe získaná) data.

K~prověření metod testování hypotéz mohou dobře posloužit také některé webové stránky, například
\href{http://statisticsonweb.tf.czu.cz/}{\tt http://statisticsonweb.tf.czu.cz}, viz \cite{6}.

Někdy býváme postaveni před rozhodnutí, které programy bychom měli pro výuku zvolit. Patrně příliš nezáleží na tom, pro který
program se rozhodneme a~který budeme při výuce používat. Studenti si dovedou zpravidla k~ostatním
programům najít cestu poměrně snadno. 

Zdeněk Pinc v~\cite{8} na str. 69 uvádí: \uv{%
Na dotaz amerických univerzit, které programovací jazyky by měli své studenty učit, aby zvýšili jejich šanci na uplatnění u~firmy IBM,
firma odpovídala, to je úplně lhostejné, nebudete\z li nám posílat hloupé, tj. uzavřené, zablokované absolventy, potřebný jazyk se naučí během pár měsíců,
který to bude, lze obtížně predikovat, patrně některý z~těch, které ještě neexistují...}

Podíváme se nyní na dvě neparametrické metody a~přiblížíme si, jak si s~nimi poradí oblíbený program Statistica 12 od firmy Statsoft. Ukážeme si na příkladech malých souborů dat, co vlastně tento program počítá, podle jakých vzorců a~jaké jsou u~těchto metod výstupy.

Neparametrické metody fungují na základě \uv{pořadových statistik}. Nevyžadují normalitu dat, předpokládají pouze spojitost
příslušných distribučních funkcí.

\newpage
\section{Wilcoxonův dvouvýběrový test}

Představíme si nejprve Wilcoxonův\footnote{Frank Wilcoxon (1892\,--\,1965), irsko-americký chemik a~statistik.}
 dvouvýběrový test (Wilcoxon rank-sum test), ekvivalentně formulovaný také pod názvem
Mann-Whitneyův test. Jde o~dvouvýběrový test, porovnává tedy dva soubory dat, podobně jako další takové testy,
například Studentův $t$ test, Aspinové-Welchův test, Cochran\z Coxův test, Kolmogorov-Smirnovův test či Wald-Wolfowitzův test.

Uvažujme dva nezávislé výběry $X_{1},\dots,X_{m}$  a~$Y_{1},\dots,Y_{n}$ ze dvou spojitých rozdělení.
Wilcoxonův dvouvýběrový test nám umožňuje rozhodnout o~shodnosti distribuční funkce těchto dvou rozdělení.
Popíšeme si postup, viz třeba 
%\cite{1}, \cite{1}
\cite{1,2,9}. Přiřadíme nejprve všem hodnotám $X_1,\dots, X_m, Y_1,\dots,Y_n$ pořadí nebo v~případě stejných hodnot průměr pořadí. Dále označíme $T_1=$ součet pořadí~$X_k$ a~$T_2=$ součet pořadí~$Y_k$ a~vypočítáme výslednou statistiku
\begin{gather*}
U=\min(U_1,U_2),\\
\mbox{kde} \quad U_1=mn+\frac12m(m+1) -T_1 \quad  \mbox{a} \quad U_2=mn+\frac12n(n+1) -T_2.
\end{gather*}

Nulovou hypotézu o~rovnosti rozdělení $X_k$ a~$Y_k$ zamítneme (s~rizikem chyby prvního druhu $\alpha$), právě když bude statistika
$U$ menší nebo rovna kritické hodnotě $W(\alpha)$, kterou zjistíme v~příslušných tabulkách (např. v~\cite{1} na str. 335).

 Pro větší rozsahy $m$, $n$ můžeme použít také aproximace
$$ Z=\frac{|U_1-\frac12 mn|}{\sqrt{\frac1{12}mn(m+n+1)}}\approx N(0;1) .$$
Nulovou hypotézu pak zamítneme, právě když $  Z\geq u(1-\tfrac{\alpha}2)$,
kde $u$ značí kvantil normálního rozdělení $N(0 ;1)$.

Podobně jako mnoho neparametrických metod vyžaduje tento test spojitost distribuční funkce. Pokud nemůžeme tuto vlastnost zaručit, lze provést jistou úpravu -- tzv. {\bf korekci na spojitost}, viz například \cite{3}. Spočívá v~tom, že čitatel ve výše uvedeném vzorci zmenšíme o~$\frac12$, vypočítáme tedy statistiku
$$Z=\frac{|U_1-\frac12mn|-\frac12}{\sqrt{\frac1{12}mn(m+n+1)}} .$$
Program Statistica umí na požádání poskytnout také tuto hodnotu.

Navíc najdeme ve výsledcích programu také jakousi upravenou hodnotu, při které uplatňujeme  tzv. {\bf korekci na shody}, viz opět například \cite{3}.
Jestliže se ve zkoumaném souboru dat vyskytují skupiny shodných hodnot, statistiku~$Z$ vydělíme konstantou
$$ C_Z=\sqrt{1-\frac{\sum_{k=1}^s (t_k^3-t_k)}{(m+n)^3-(m+n)}} , $$
kde $t_1,\dots , t_s$ jsou rozsahy skupin stejných hodnot veličin $X_k$ a~$Y_k$. Pro statistiku $Z$ (ať už u~ní provádíme korekci na spojitost, nebo ne)
počítáme tedy hodnotu upravenou s~ohledem na korekci na shody
$$ Z_{adj}=\frac{Z}{C_{Z}}. $$

Přiblížíme si vše na následujícím příkladu.

\subsection*{Příklad 1}
Máme porovnat spotřebu (v~l/100 km) u~dvou typů automobilů Citroen C3 a~BMW 7 na základě záznamů dvanácti řidičů:

\begin{table}[!ht]
\centering
\begin{tabular}{lccccccccccc}
Citroen C3  & 6,3 & 6,5 & 7,0 & 6,3 & 6,4 &     & \\[3pt]
BMW 7       & 6,0 & 6,1 & 6,3 & 6,6 & 5,7 & 6,1 & 5,9
\end{tabular}
\end{table}

Budeme postupovat podle návodu. Máme $m=5$ a~$n=7$. Přiřadíme datům z~první i~druhé skupiny dohromady pořadí:

\begin{table}[!ht]
\centering
\begin{tabular}{lccccccccccc}
Citroen C3  & 6,3 & 6,5 & 7,0 & 6,3 & 6,4 &     & \\
Pořadí      & 7   &  10 &  12 &  7  &  9  &     & \\
\hline
BMW 7       & 6,0 & 6,1 & 6,3 & 6,6 & 5,7 & 6,1 & 5,9 \\
Pořadí      & 3   & 4,5 &  7  & 11  &  1  & 4,5 & 2
\end{tabular}
\end{table}

A~vypočítáme součty pořadí $T_1=45$, $T_2=33$ a~statistiky
$$U_1=5\cdot 7 + \frac12 \cdot 5\cdot 6 -33=5 , \
U_2=5\cdot 7 + \frac12 \cdot 7\cdot 8 -45=30 \ \mbox{a} \ U=\min(U_1,U_2)=5 .$$

Protože $U=5$ je menší nebo rovno mezní hodnotě $W(0{,}05)=5$ (např. v~\cite{1} na str. 335), shodnost rozdělení těchto dvou skupin veličin můžeme zamítnout na hladině $\alpha = 0{,}05$.

Pro výpočet pomocí programu Statistica upravíme data následujícícm způsobem. Zahrneme spotřebu obou typů automobilů do jednoho sloupce a~ve vedlejším sloupci provedeme rozlišení třeba pomocí číslic 1 a~2. K~Wilcoxonovu dvouvýběrovému testu získáme přístup prostřednictvím záložek {\it Statistiky}, {\it Neparametrické statistiky}, {\it Porovnání dvou nezávislých vzorků (skupiny)} a~{\it Mann-Whitneyův \b{U}-test}, vše vidíme na  \hyperref[obrazek1]{obrázku 1}.

\begin{figure}[!ht]
\fbox{\includegraphics[width=\delkas]{WIL1.jpg}}%\\
\caption{Zadání Wilcoxonova testu v~programu Statistica.}
\label{obrazek1}
\end{figure}

Ve výsledcích programu Statistica vidíme kromě součtů $T_1$, $T_2$ a~statistiky~$U$ také příslušnou $p$-hodnotu $p=0{,}047980$, viz \hyperref[obrazek2]{obrázek 2}.

\begin{figure}[!hbt]
\fbox{\includegraphics[width=\delkas]{WIL2.jpg}}%\\
\caption{Výsledky Wilcoxonova testu bez korekce na spojitost.}
\label{obrazek2}
\end{figure}

Shodnost spotřeby u~obou typů automobilů můžeme tedy zamítnout s~rizikem chyby prvního druhu cca 4,8\,\%.

Program Statistica však obsahuje také aproximaci
$$ Z=5\cdot \sqrt{ \frac{15}{91}} =2{,}029994857 $$
a~příslušnou $p$-hodnotu $p=0{,}042358$.

Pokud navíc uplatníme korekci na shody, obdržíme upravenou statistiku
$$ Z_{adj}=5\cdot \sqrt{ \frac{15}{91}}\cdot \sqrt{\frac{286}{281}} =2{,}047975677 $$
a~příslušnou $p$-hodnotu $p=0{,}040563$.

Pro výpočet $C_Z$ jsme si uvědomili, že v~našem příkladu máme dvě skupiny stejných veličin,  $t_1=2$ (dvě hodnoty 6,1)  a~$t_2=3$ (tři hodnoty 6,3), a~proto dostáváme konstantu
$$C_Z=\sqrt{1-\frac{(2^3-2) + (3^3-3)}{12^3-12}}=\sqrt{\frac{281}{286}} .$$

Na \hyperref[obrazek1]{obrázku 1} vidíme také políčko, jehož zaškrtnutím lze zadat pro Wilcoxonův (Mann-Whitneyův) test volbu provedení korekce na spojitost.


Pak dostaneme trošku jiné aproximované statistiky
  $$Z=24\cdot \sqrt{ \frac{3}{455}} =1{,}948795063 $$
  s~příslušnou $p$-hodnotou $p=0{,}051321$. Uplatníme\z li navíc korekci na shody, obdržíme upravenou statistiku
  $$Z_{adj}=24\cdot \sqrt{ \frac{3}{455}}\cdot \sqrt{\frac{286}{281}} =1{,}96605665 $$
  s~příslušnou $p$-hodnotou $p=0{,}049293$, viz \hyperref[obrazek3]{obrázek 3}.

\begin{figure}[!bht]
\fbox{\includegraphics[width=\delkas]{WIL3.jpg}}%\\
\caption{Výsledky Wilcoxonova testu s~korekcí na spojitost.}
\label{obrazek3}
\end{figure}

 Poznamenejme, že pro malé rozsahy výběrů nejsou poslední čtyři statistiky vhodné, pro tento příklad jsme je uvedli pouze na ukázku toho, jaké výpočty provádí program Statistica.


\section{Spearmanův korelační koeficient}

Pro vyjádření závislosti či nezávislosti veličin počítáme zpravidla korelační koeficienty. Kromě Pearsonova nebo Kendallova koeficientu užíváme neparametrickou metodu založenou
na tzv. Spearmanovu\footnote{Charles Spearman (1863\,--\,1945), anglický psycholog a~statistik.}
 korelačním koeficientu.

Uvažujme náhodný výběr $(X_{1},Y_{1}),\dots,(X_{n},Y_{n})$ ze spojitého dvourozměrného rozdělení. Chceme testovat, že náhodné veličiny $X_k$ a $Y_k$ jsou nezávislé.
Nejprve přiřadíme  hodnotám $X_1,\dots, X_n$ pořadí $Q_1,\dots, Q_n$ a
 hodnotám $Y_1,\dots, Y_n$  pořadí $R_1,\dots, R_n$. Spearmanův korelační koeficient počítáme zpravidla podle vzorce
$$ r_S=1-\frac6{n(n^2-1)}\cdot \sum_{k=1}n (R_k-Q_k)^2 ,$$
viz například \cite{1,2,9}.

 Nulovou hypotézu o~nezávislosti $X_k$ a~$Y_k$  zamítneme (na hladině $\alpha$), právě když
$ |r_S| $ překročí příslušnou kritickou mez $r_S(\alpha)$, kterou najdeme v~tabulkách (např. v~\cite{1} na str. 330).

Programu Statistica ale užívá jiný vzorec pro tento koeficient. Při tomto postupu vlastně uplatníme běžný Pearsonův korelační koeficient na pořadí $Q_1,\dots, Q_n$ a~$R_1,\dots, R_n$, tedy
$$r_{S, adj}= r_{Q,R}=\frac{S_{Q,R}}{\sqrt{ S_Q^2 \cdot S_R^2}} ,$$
viz například \cite{3}.
Pokud v~souborech nejsou shodné hodnoty, oba postupy dávají stejný výsledek.

 Ukážeme si opět na následujícím příkladu.


\subsection*{Příklad 2}

Máme k~dispozici údaje o~průměrném znečištění ovzduší oxidy síry (v~{$\mu$}g/m{$^3$}) v~sedmi oblastech a~o~výskytu dětské alergie (v~\%) v~těchto regionech:

\begin{table}[!ht]
\centering
\begin{tabular}{l *{7}{r} }
Znečištění  & 320 & 240 & 130 & 360 & 210 & 200 & 150 \\
Alergie     & \phantom3 45  &  \phantom2 30 &  \phantom1 25
  &  \phantom3 40 & \phantom2 25  &  \phantom2 20 & \phantom1 25
\end{tabular}
\end{table}

Souvisejí spolu tyto dvě veličiny?

Počet měření je $n=7$. Přiřadíme každé skupině dat příslušná pořadí:

\begin{table}[!ht]
\centering
\begin{tabular}{lccccccccccc}
Znečištění  & 320 & 240 & 130 & 360 & 210 & 200 & 150 \\
Pořadí      &  6  &  5  &  1  &  7  &  4  &  3  &  2  \\[2pt]
\hline
Alergie     & 45  &  30 &  25 &  40 & 25  &  20 & 25  \\
Pořadí      &  7  &  5  &  3  &  6  &  3  &  1  &  3
\end{tabular}
\end{table}

Odtud vypočítáme Spearmanův korelační koeficient (podle uvedeného vzorce)
$$r_S=1-\frac{6}{7\cdot 48}\cdot \left(1^2+0^2+2^2+1^2+1^2+2^2+1^2 \right)=\frac{11}{14}= 0{,}7857.$$
Protože absolutní hodnota z~$r_S$ překročí kritickou hodnotu $r_S(0{,}05)=0{,}7450$ (viz \cite{1}, str. 330),
nezávislost veličin zamítáme. Můžeme tedy prohlásit, že (s~rizikem chyby prvního druhu 5\,\%) uvedené veličiny spolu souvisejí.

Program Statistica provádí alternativní výpočet Spearmanova korelačního koeficientu
$$
S_Q^2= \frac{14}{3}, \quad S_R^2=\frac{13}{3} , \quad S_{Q,R}=\frac{7}{2} \quad \mbox{a} \quad r_{S,adj} =\frac32\sqrt{\frac{7}{26}}=0{,}7783117825 .
$$

\begin{figure}[!hbt]
\centering
\fbox{\includegraphics[width=\delkas]{SPEAR.jpg}}%\\
\caption{Zadání a~výsledek pro Spearmanův korelační koeficient.}
\label{obrazek4}
\end{figure}


Přístup k~tomuto korelačnímu koeficientu provádíme opět pomocí záložek
{\it Statistiky}, {\it Neparametrické statistiky}, {\it Korelace (Spearmanův...)} a~{\it S\b{p}earmanův koef. R}. Zadání i~výsledek vidíme na \hyperref[obrazek4]{obrázku 4}.


\newpage
\section*{Závěrem}

Je známo, že zotročování je možné provádět velmi účinně nejen za pomoci nejrůznějších prvků násilí, byrokracie, cenzury, 
ale také prostřednictvím rozsáhlé a~masivní nabídky.
Počítačové programy přinášejí často nepřeberné množství výsledků, které se někdy mohou lišit od výsledků uvedených v~některých
učebnicích nebo se v~nich vůbec nevyskytují.
Taková situace vzbuzuje u~studentů a~jiných uživatelů jisté rozpaky. Máme sice mnoho čísel či grafů, ale nevíme co znamenají. Výhoda snadných výpočtů je zkalena nesrozumitelností výsledků.

V~orientaci nepomáhá většinou ani dokumentace programů. Mnohé informace a~první nasměrování lze často vyhledat na internetu. Nalezené postupy je pak nutné 
ověřit a vyzkoušet v~příslušných programech. Následně uživatel zjistí, že ve svém úsilí nebyl sám a~nebyl první, kdo se
pokoušel podobné nesrovnalosti řešit. Potřebné vzorce většinou nakonec dohledá v~literatuře nebo se o~nich dozví od kolegů. 

Dvojí způsob výpočtu Spearmanova korelačního koeficientu je
popsán v~mnoha (byť ne ve všech) učebnicích, např. 
\cite{3}, str.\,257, 
\cite{4}, str.\,778, 
\cite{5}, str.\,34, 
\cite{7}, str.\,134, 
\cite{10}, str.\,92 a~%
\cite{11}, str.\,153.

Naopak zmínky o~korekcích u~Wilcoxonova testu se vyskytují poměrně zřídka, např. 
\cite{5}, str.\,25, 
\cite{7}, str.\,134, 
\cite{10}, str.\,150 a~%
\cite{11}, str.\,102.

Porozumění tomu, co vlastně počítačové programy počítají a~poskytují, přináší jisté osvobození z~otroctví nabídky a~tyto
nástroje se pak stanou dobrým pomocníkem.


\renewcommand{\refname}{Literatura}
\begin{thebibliography}{10}
\setlength\itemsep{2pt}
\selectlanguage{english}
\frenchspacing
\bibitem{1} Anděl, J. (1985): {\it Matematická statistika.} SNTL Praha, Alfa Bratislava, 1985.

\bibitem{2} Anděl, J. (2007): {\it Statistické metody.} Matfyzpress Praha, 2007.

\bibitem{3} Anděl, J. (2007): {\it Základy matematické statistiky.} Matfyzpress Praha, 2007.

\bibitem{4} Meloun, M., Militký, J. (2004): {\it Statistická analýza experimentálních dat.} Academia Praha, 2007.

\bibitem{5} Mošna, F. (2017): {\it Základní statistické metody.} PedF UK Praha, 2017.

\bibitem{6} Mošna, F., Lubanda, D.\,M. (2015): Webové stránky pro testování hypotéz. {\it Matematika-Fyzika-Informatika} {\bf 24}(5), 371--377.

\bibitem{7} Pecáková, J. (2011): {\it Statistika v terénních průzkumech.} Professional Publishing Praha, 2011.

\bibitem{8} Pinc, Z. (1999): {\it Fragmenty k filosofii výchovy.} Oikoymenh Praha, 1999.

\bibitem{9} Reif, J. (2000): {\it Metody matematické statistiky.} Západočeská univerzita Plzeň, 2000.

\bibitem{10} Řezanková, H. (2010): {\it Analýza dat z dotazníkových šetření.} Professional Publishing Praha, 2010.

\bibitem{11} Zvára, K. (2004): {\it Biostatistika.} Matfyzpress Praha, 2004.

\end{thebibliography}



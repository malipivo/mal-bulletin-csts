% !TEX TS-program = LuaLaTeX
% !TEX encoding = UTF-8 Unicode 
% !TEX root = ../../!hlavni-soubor.tex

\def\zvyrazni#1{\textit{#1}}
% zdroj:
% https://docs.google.com/spreadsheets/d/1f62JXcEKNng9hPPy0CMj3gtEOV3jyJn_D0CVPAwDk5M/edit?usp=sharing
\def\maljadro{\hfill}% $\qed$
\def\malskok{\maljadro\\[-3pt]}


\gdef\mujnazevCS{Inspirace STAtistických KANtorů}
\gdef\mujnazevEN{Inspirations from teachers of statistics}
\gdef\mujnazevPR{\mujnazevCS}
\gdef\mujnazevDR{\mujnazevEN}
\gdef\mujauthor{Ondřej Vencálek}

\bonushyper

\nazev{\mujnazevCS}

\nazev{\mujnazevEN}

\author{\mujauthor}

%\Adresa{$^1$Katedra matematické analýzy a~aplikací matematiky, Přírodovědecká fakulta Univerzity Palackého, 17. listopadu 12, 771\,46 Olomouc}

\Email{ondrej.vencalek@upol.cz}

%\Abstrakt{}
%\smallskip
%\KlicovaSlova{}

%\Abstract{}
%\smallskip
%\KeyWords{}
\bigskip

\noindent
V rámci konference STAKAN 2019 proběhla mezi účastníky anketa, ve které se organizátoři ptali na čtyři otázky týkající se výuky statistiky na vysokých a~středních školách. Ankety se zúčastnilo devět odvážných STAtistických KANtorů. Věříme, že v jejich odpovědích na první dvě otázky čtenáři Informačního bulletinu naleznou inspirativní zdroje nejen pro přípravu přednášek, ale také pro další sebevzdělávání. Druhé dvě otázky zase vyzývají k úvahám o~popularizaci statistiky a~o~obsahu kurzů statistiky na středních školách.   

\section{Kniha, která Vás ovlivnila při přípravě kurzu týkajícího se statistiky}
\smallskip

\begin{itemize}
\itemsep=-2pt
\item Anderson: \zvyrazni{Statistics for Economics} zrozumiteľne, systematicky členená.%
\malskok

\item Andy Field: \zvyrazni{Discovering Statistics using SPSS}. Perfektní kniha, která zdaleka není jen manuálem na SPSS, naopak jde po věcné podstatě. Je to skvělá učebnice statistiky, navíc užitečně členěná podle stupňů obtížnosti. Vřele doporučuji statistikům a~hlavně nestatistikům.
\malskok

\item Groebner, Shannon, Fry, Smith: \zvyrazni{Business Statistics. A~Decision Making Approach}. 
Pearson, Prentice Hall, 7. vydání, 19 kapitol, 1040~stran (aktuální je 10. vydání). Věcně podaná a~stručná teorie s~předpoklady, jednotná grafika, rozumné barevné rozlišení (výklad, vzorce, příklady apod.).  Ukázkový příklad s~vysvětleným řešením ke každému typu statistického problému. Příklady k~samostatnému procvičení (koncepční i~prakticky zaměřené problémy). Na CD i~na webu jsou data k~příkladům, návody pro Excel a~Minitab, program PHStat2, prezentace ke každé kapitole a~self-testy ke každé kapitole  (na webu byly volně dostupné, nyní už skryté cca od roku 2015, na přístup odkazuji z~vlastního kurzu).
\item Gareth J., Witten D., Hastie T., and Tibshirani R.: \zvyrazni{An Introduction to Statistical Learning with Applications in R.} \href{http://faculty.marshall.usc.edu/gareth-james/}{\url{http://faculty.marshall.usc.edu/gareth-james/}}. Výborná učebnice. Statistické modely s~aplikacemi, text knihy, data i~kódy R jsou na webu. Příklady jsou doplněné vysvětlením základních statistických principů. Hodí se pro pokročilejší kurzy nebo pro zájemce o~rozšíření znalostí. 
\malskok

\item James T. McClave, P. George Benson, Terry T. Sincich: \zvyrazni{Statistics for Business and Economics}. 864~stran, ISBN 032182623X. Teraz je už 13. vydanie a~pred 20 rokmi, kedy som začala učiť na FM UK, to bolo len 2. vydanie. Ku knihe boli dostupne aj PPT prezentácie a~ja som si vtedy podľa tejto knihy urobila moje prezentácie v~slovenčine a~používam ich dodnes ako základ kurzu štatistiky na FM UK pre 2. ročník Bc. V~knihe je teória štatistiky vysvetlená veľmi jednoducho a~zrozumiteľne na príkladoch z~ekonómie a~manažmentu.
\malskok

\item David MacKay: \zvyrazni{Information Theory, Inference, and Learning Algorithms}.
\item Chris Bishop: \zvyrazni{Pattern Recognition and Machine Learning}.
\malskok

\item Helmuth Swoboda: \zvyrazni{Moderní statistika} vhodné pro pro SŠ.
\item Robová, Hála, Calda: \zvyrazni{Matematika pro SŠ -- Komplexní čísla, kombinatorika, pravděpodobnost a~statistika}. Pěkná učebnice pro SŠ.
\item Jiří Reif: \zvyrazni{Metody matematické statistiky}.
\item Charamza, Hanousek: \zvyrazni{Moderní metody zpracování dat}.
\item Učebnice J. Anděla, J. Štěpána, J. Machka\ldots
\malskok

\item Lepš, Šmilauer: \zvyrazni{Biostatistika}. \href{https://www.martinus.sk/?uItem=248730}{\url{www.martinus.sk/?uItem=248730}} a~ich prednášky a~príklady.
\item Motulsky: \zvyrazni{Intuitive Biostatistics: A~Nonmathematical Guide to Statistical Thinking}. \href{https://www.bookdepository.com/Intuitive-Biostatistics-Harvey-Motulsky/9780199946648}{\url{www.bookdepository.com/Intuitive-Biostatistics-} \url{Harvey-Motulsky/9780199946648}}.
\item Roberts Russo: \zvyrazni{A~student's guide to analysis of variance} \href{https://www.amazon.com/Students-Guide-Analysis-Variance/dp/0415165652}{\url{https://www.amazon.com/Students-Guide-Analysis-Variance/dp/0415165652}}.
\item Pekár, Brabec: \zvyrazni{Moderní analýza biologických dat I~a~II}.
\malskok

\item J. Rosenthal: \zvyrazni{Zasažen bleskem}.
\item H. Rosling: \zvyrazni{Faktomluva}.
\malskok

\item Michal Čihák: \zvyrazni{Výuka pravděpodobnosti na gymnáziu s~využitím počítačů}. Měl jsem tu čest připravovat knižní vydání z~obhájené disertační práce (školitel prof. Antoch). Bylo to inspirativní z~pohledu volby příkladů z~oblasti pravděpodobnosti, ale i~sazby v~\TeX u. Byla to, myslím, nejsem si jistý na sto procent, první kniha, u~které jsem zahlédl možnost po nakliknutí otevření zdrojového kódu v~dalším programu a~jeho spuštění. Měl jsem tu čest se s~autorem knihy setkat osobně.
\maljadro
\end{itemize}


\section{Video na youtube či jinde, online výukové materiály, případně online kurz -- prostě to, co používáte při výuce, či Vás inspirovalo při přípravě výuky}
\smallskip

\begin{itemize}
\itemsep=-2pt
\item \href{https://www.khanacademy.org/math/statistics-probability}{Khan Academy} Basics of Statistics.
\malskok

\item \href{https://www.youtube.com/watch?v=GczkFbi4ezM}{\url{https://www.youtube.com/watch?v=GczkFbi4ezM}} Skvělá přednáška prof. Cyrila Höschla. Není statistikem, ale o~statistice mluví krásně.%
\malskok

\item Skripta \zvyrazni{Matematická biologie} dostupné zde: \href{http://www.matematickabiologie.cz/index.php?pg=pro-studenty--vyukove-materialy}{\url{http://www.matematicka} \url{biologie.cz/index.php?pg=pro-studenty--vyukove-materialy}}
\malskok

\item Odkazy na užitečné informace k~IBM SPSS Statistics,
\href{https://www.lib.sfu.ca/find/research-tools/spss-resources}{\url{https://www.lib.sfu.ca/find/research-tools/spss-resources}}
\item Jeden z užitečných návodů k R: \zvyrazni{R user Anthony Damico has created ``Twotorials":} a series of two-minute tutorials for newcomers to R.
\href{https://www.youtube.com/playlist?list=PLcgz5kNZFCkzSyBG3H-rUaPHoBXgijHfC}{\small\url{www.youtube.com/playlist?list=PLcgz5kNZFCkzSyBG3H-rUaPHoBXgijHfC}}
\malskok

\item Juraj Hromkovič -- škola budúcnosti: \href{https://www.youtube.com/watch?v=AMZnKylTQv4}{\url{https://www.youtube.com/}}\\ \href{https://www.youtube.com/watch?v=AMZnKylTQv4}{\url{watch?v=AMZnKylTQv4}}
\item YouTube: \href{https://www.youtube.com/user/professorleonard57}{Professor Leonard} Statistics Lecture.
\item YouTube: \href{https://www.youtube.com/results?search_query=How+to+Find+the+Sample+Size+Statistics}{How to Find the Sample Size Statistics}.
\malskok

\item Andrew Ng: Machine Learning, \href{https://www.coursera.org/}{\url{https://www.coursera.org/}}.
\item Idan Segev: Synapses, Neurons and Brains, \href{https://www.coursera.org/}{\url{www.coursera.org/}}.
\item David MacKay lectures, \href{http://videolectures.net/}{\url{http://videolectures.net/}}.
\malskok

\item \href{https://www.wessa.net/test.wasp}{\url{https://www.wessa.net/test.wasp}}
\item \href{https://www.mathportal.org/calculators/statistics-calculator/}{\url{https://www.mathportal.org/calculators/statistics-calculator/}}
\item \href{https://www.graphpad.com/quickcalcs/}{\url{https://www.graphpad.com/quickcalcs/}}
\item \href{https://www.socscistatistics.com/tests/}{\url{https://www.socscistatistics.com/tests/}}
\item \href{https://www.IntellectusStatistics.com}{\url{https://www.IntellectusStatistics.com}}
\item Odkazy na další stránky jsou na \href{http://statpages.info}{\url{http://statpages.info}}.
\item \href{http://statisticsonweb.tf.czu.cz}{\url{http://statisticsonweb.tf.czu.cz}}
\malskok

\item Motulsky, Christopoulos: \href{https://www.facm.ucl.ac.be/intranet/books/statistics/Prism-Regression-Book.unlocked.pdf}{Fitting Models to Biological Data Using Linear and Nonlinear Regression} (on-line).
\malskok

\item \href{https://www.datacamp.com/}{\url{www.datacamp.com}} Možnost umožnit studentům přístup k~online kurzům zdarma.
\malskok

\item Hltám \href{https://www.youtube.com/user/shiffman}{The Coding Train} na YouTube. Inspirovalo to ke studiu Java\-Scriptu a~i~k~ponoření se do zajímavých příkladů. Shiffman mě dostal převodem z~C++ do JavaScriptu u~\href{https://www.youtube.com/watch?v=alhpH6ECFvQ}{simulace toku kapalin}. Hned si člověk musel vyzkoušet \href{https://obsproject.com/}{Open Broadcaster Software alias OBS}, který užívá při streamování. Tyto dny koukám na TV, na \href{https://www.twitch.tv/}{\tt twitch.tv}. Jak a~v~čem lidé programují (Mathematica, Python, Ruby ap.), hackují (CTF, TryHackMe), jak si tvoří nádstavby přímo pro Twitch, např. vykřičník a~termín v~chatu vykoná to či ono: pustí písničku na přání, zobrazí skóre u~šachových turnajů, seznam parametrů jako u~{\tt--help} v~Linuxu ad.
\maljadro
\end{itemize}



\section{Jak popularizovat statistiku na středních školách a~pro širší veřejnost?}
\smallskip

\begin{itemize}
\item Jezdit přímo na školy s~odbornými přednáškami.
\item Nabídnout středoškolským učitelům popularizační semináře statistiky, které středoškolákům ukáží praktické využití statistiky.
\item Nenásilně. Snažíme se, v~prostorách fakulty mám tabule s~řešenými ukázkami příkladů: Bayes, simulační úloha -- bootstrap odhad mezí spolehlivosti, testy apod.
\item … mám informáciu, že v~Poľsku organizujú olympiádu zo štatistiky pre študentov stredných škôl. Organizuje to Poľská štatistická spoločnosť.
\item Nejlepší je fakt asi ty školy objíždět s~popularizačními přednáškami.
\item Asi je třeba přesvědčit žáky i~veřejnost o~užitečnosti statistiky. Možná by mohli novináři a~vůbec všichni, co něco publikují, více používat statistické nástroje a~ukázat, že tabulka nebo graf vypoví často daleko více, přesněji a~efektivněji o~situaci či problému než spousta slov. Statistiku lze užívat ve škole v~mnoha předmětech nejen v~matematice. 
\item Přes školu hrou jako za Komenského. Ať už šachy či go (rozhodování při úplných informacích) nebo poker či bridž (rozhodování při neúplných informacích). Bridž učí v~Karviné a~naši nejlepší hráči jsou jejich zásluha. Snažil bych se co nejvíc šířit povědomí o~programování v~duchu \uv{Udělej si své dítě}, určitě Linux, open software, open hardware, open data a~tato cesta něco změřit, něco si ochytat. Už neučím, ale chodil bych co nejvíc ven z~budovy: na exkurze, na výlety, klidně s~nimi řešil chod přírody někde na zahradě u~grilu u zlatého řezu\ldots
\end{itemize}



\section{Co by, podle Vás, měl maturant vědět z~oblasti pravděpodobnosti a~statistiky? Jakým konceptům by měl rozumět a~jaké techniky by měl umět aktivně použít?}
\smallskip

\enlargethispage{-1.5\baselineskip}
\begin{itemize}
%\itemsep=-2pt
\item Popisná štatistika -- čítanie grafov, kvantitatívne charakteristiky a~naj\-mä základné kompetencie v~oblasti matematiky (nesmú sa obzerať, keď poviem: \uv{Píšte do čitateľa.} a~poznať grécku abecedu do tej miery, že psí nie je nadávka).
\item Neměl by již ze střední školy mít ke statistice a~matematice odpor. To by podle mého názoru bylo zcela dostačující.
\item Podle mé zkušenosti začne statistika studenty zajímat až ve chvíli, kdy ji potřebují. Když řeší vlastní problém a~mají vlastní data, potom uvažují co s~nimi. U~nás projdou dvěma kurzy statistiky a~přesto někteří nedovedou použít v~diplomové práci jednoduché poznatky, ze kterých měli zkoušku a~přijdou se poradit. Zdůrazňuji, že někteří. Ohledně střední školy nebo nižších stupňů: Některé státy dbají na znalost metod popisu světa a~mají vypracovaný systém statistických pojmů, které se mají děti postupně učit. V~našich podmínkách by se měl nejdřív inventarizovat celý obsah výuky a~požadovaných znalostí na základních a~středních školách a~sjednotit je, vytvořit povinné minimum. Potom sestavit dal\-ší priority a~navrhnout, do kterých ročníků je zařadit. Do současného systému bych základy statistiky (popisné statistiky a~grafy, případně odhady \uv{od oka}) dala jen volitelně. Možná pro časovou řadu bych preferovala pojem srovnatelnosti dat, kdy je graf časové řady užitečný. Prognóza je hodně složitý problém, který ani statistika neumí přesně řešit. Někteří studenti znají základy statistiky a~počtu pravděpodobnosti z~gymnázia a~určitě to při další výuce pomáhá.

Odkazy například: core standards\\
\href{http://www.corestandards.org/Math/Content/7/RP/A/2/d/}{\url{http://www.corestandards.org/Math/Content/7/RP/A/2/d/}} a\\
\href{http://www.corestandards.org/read-the-standards/}{\url{http://www.corestandards.org/read-the-standards/}}

A~doplněk přípravy učitelů: National Council of teachers of mathematics
\href{https://www.khanacademy.org/math/statistics-probability/}{\url{www.khanacademy.org/math/statistics-probability/}}.
V~české verzi -- detaily ani rozsah neznám:
\href{https://khanovaskola.cz/}{\url{https://khanovaskola.cz/}}.

\item Podľa mňa stačí, keď sa na SŠ študent naučí popisnú štatistiku, čiže dáta popísať základnými štatistikami a~grafmi v~rozsahu čo obsahuje Excel a~aj to ovládať urobiť v~Exceli a~nie ručne. Ďalej by ich mali naučiť používať údaje, ktoré sú na stránkach ŠU a~Eurostatu a~dať im urobiť nejaký zmysluplný projekt z~týchto údajov s~interpretáciami, čiže učiť ich niečo praktické a~zaujímavé\ldots{} Tiež sa musí naučiť základy pravdepodobnosti a~rozumieť na čo to je. Hlavne by im učitelia mali vysvetliť, že dnes je doba big data a~z~tých údajov sa dajú získať cenné informácie pre rozhodovanie rôznymi matematicko-štatistickými metódami a~metódami AI\ldots{} ľudia, ktorí to vedia robiť, sú žiadaní a~ich hodnota bude stále rásť, preto sa treba učiť aj matematiku aj štatistiku a~ísť toto ďalej študovať (data science)\ldots{}

\item[]
\leftskip4ex
\makebox[0pt][r]{$\bullet$\hskip5ex}%
\makebox[0pt][r]{1. }Měl by umět nějak vizualizovat data a~číst z~různých grafů ($=$ interpretovat je).\\
\makebox[0pt][r]{2. }Měl by chápat pojem pravděpodobnosti a~nejistoty.\\
\makebox[0pt][r]{3. }Měl by chápat, co je predikce a~měl by mít vyzkoušeno, jak se pozná, jestli je predikce dobrá.\\
\makebox[0pt][r]{4. }Měl by intuitivně chápat overfitting.\\
\makebox[0pt][r]{5. }Neměl by umět taxonomii variace, kombinace, permutace, s~opakováním a~bez. Je to matoucí a~zbytečné.

\leftskip0ex
\item Měl by umět vyjadřovat skutečnosti pomocí statistických nástrojů a~pojmů, rozumět významu grafů, charakteristik, nezávislosti a~zejména rozmýšlet o~smyslu zákona velkých čísel, limitních vět a~vůbec o~náhodnosti...

\item Statistika: popisná statistika, interpretace popisných statistik a~vizualizací, maturanti by měli znát základní způsoby manipulace s~grafy (změna měřítka, ořezání osy, 3D grafy, \ldots{}).

\item S~úsměvem píši, že by mi měl předložit zprávu z~exkurze za každou velkou oblast lidského bádání: lékařství / medicína, strojařina, typografie, počítačová grafika / animace / film, letecký či kosmický průmysl, topografie / tvorba map. Něco takového mimo rámec středoškolských osnov. Něco v~duchu seriálu \uv{Jak se co vyrábí}, ale ošahané naživo.

\end{itemize}


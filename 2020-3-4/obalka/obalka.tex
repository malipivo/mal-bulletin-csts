
% an envelope of the issue

% !TEX TS-program = lualatex
\newcount\moje
%\input soubor.tex
\moje=0
\def\xbonus{-\par}
\def\prostor{} % new

\newcount\malosnova
\input malosnova.tex


\documentclass{article}
\usepackage[ngerman,english,slovak,czech]{babel}
\usepackage[utf8]{luainputenc}
\usepackage[IL2]{fontenc}
\usepackage{tikz}
\usepackage[colorlinks,allcolors=black]{hyperref}

\def\CS{$\cal C\kern-.1667em\lower.5ex\hbox{$\cal S$}\kern-.075em $}
\def\CSfonts{\CS\kern.1em fonts}

\renewcommand\rmdefault{cmr}
\renewcommand\sfdefault{cmss}
\renewcommand\ttdefault{cmtt}


\newcount\cislo
%\cislo=2
\newcount\rocnik
%\rocnik=5
\IfFileExists{../cislo.tex}{\input ../cislo.tex}{\cislo=1}

\pgfmathparse{int(\rocnik+2008)}
\let\malrok\pgfmathresult
\def\mtext{\ifcase\cislo\or březen\or červen\or září\or prosinec\fi\space \malrok}

\newcount\mrocnik
\mrocnik=\rocnik
\advance\mrocnik by 19
\newcount\citaj\citaj=0
\newcount\vyska\vyska=-472

%\mrocnik=36
%
\pgfmathparse{int(floor(\mrocnik/10))}
\let\obdelniku=\pgfmathresult
%
\pgfmathparse{int(\mrocnik-10*\obdelniku)}
\let\ctvercu=\pgfmathresult
%\typeout{\ctvercu}
%\ifcase\cislo\or
%  \vyska=-28\or
%  \vyska=-148\or
%  \vyska=-268\or
%  \vyska=-388\or\relax
%  \fi
%\font\myast=csb10 at 4mm
\def\odskoc{\\}


\usepackage{pdfpages,graphicx,color}
\definecolor{mygray}{gray}{0.5}
\pagecolor{white}

\makeatletter
\def\dotpfill{%
  \leavevmode%
  \leaders\hb@xt@.33em{\hss.\hss}\hfill%
  \kern\z@}
\makeatother 

% bylo
%\newcommand{\pr}[1]{\,\dotpfill\makebox[4.5mm][r]{\pageref{#1}}\\[4pt]}
% nové, 2020-04-22
\newcommand{\pr}[1]{\,\dotpfill\makebox[4.5mm][r]{\pageref{#1}}\\[4pt]}

%\def\zloma{\\}
%\def\zlomb#1{\pr{#1}}

%\def\uprav#1#2#3#4#5{{{#1}{#2}}}
%\def\newlabel#1#2{{#1}{\uprav{#2}}}

%\newlabel{vencalek/vencalek}{{}{1}{}{Doc-Start}{}}
%\newlabel{vencalek/vencalek}{{}{2}}


\makeatletter
\IfFileExists{../mal-core.aux}{
% This used to be !hlavni-soubor.tex.
% Preamble + TeX style + article-dependent packages.
% I am often changing \graphicspath{} and using \vloz for article inclusion at the end of this file.

\listfiles

%\nonstopmode
\newcount\zahlavi
\zahlavi=0 % 0 či 1, dát červenou pracovní verze?
%\IfFileExists{zahlavi.tex}{\input zahlavi.tex}{}
%\newcount\citachyper %\citachyper=0
%\def\thecitachyper{a}

% !TEX TS-program = lualatex
% !TEX encoding = UTF-8
\documentclass[a4paper,10pt,twoside]{article}

\usepackage[novoc]{arabluatex} % 

\IfFileExists{prace.tex}{\input prace.tex}{}
\def\pracex{ne}


\usepackage[czech,english,ngerman]{babel}
% german,germanb,slovak,
\usepackage{csbulacronym}

\usepackage{bbding}
\usepackage[utf8]{luainputenc}
\usepackage[IL2]{fontenc}
%\usepackage{amssymb}
%\usepackage{amsmath}
%\usepackage{amsfonts}
%\usepackage{amsbsy}
%\usepackage{amsthm}

%\newtheorem{theorem}{Theorem}
%\usepackage[final]{graphicx} % [draft]
%\usepackage{epstopdf}
%\usepackage{stmaryrd}
%\usepackage{fontspec}
%\usepackage{luaotfload}
%\usepackage{fontspec}

  %\usepackage{algorithmic}
  %\usepackage{algorithm}
  %\usepackage{tipa}
  %\let\oldtipa=\tipa
  %\def\tipa#1{\oldtipa{#1}}
  %\usepackage[figuresright]{rotating}
  %\usepackage{multirow}
  
\usepackage{xcolor}

%\usepackage[figuresright]{rotating}
%\usepackage{zref-savepos}
%\newcommand*{\currpos}[1]{%
  %\zsavepos{#1}%
  %%(\zposx{#1}sp, \zposy{#1}sp) =
  %%(\the\dimexpr\zposx{#1}sp\relax, \the\dimexpr\zposy{#1}sp\relax)%
%}

%\def\cellcolor#1{}
%\usepackage{psfrag, afterpage, enumerate, longtable, colortbl, } %
%\usepackage{afterpage}
%\usepackage{tabularx}
%\usepackage{longtable}

%\usepackage{multirow}
%\usepackage{colortbl}
%\usepackage{multirow,colortbl,latexsym}
\usepackage{tikz}
%\usetikzlibrary{intersections}
%\usetikzlibrary{shapes,positioning}
%\usetikzlibrary{shapes.multipart,positioning}

\usepackage{tikzlings}


%\usetikzlibrary{calc}
%\usepackage{zref-savepos}
%\usepackage{tabu}
%\usepackage{hhline}
%\usetikzlibrary{positioning}
%\usepackage{smejliky}
%\usetikzlibrary{positioning, shapes.geometric, calc}
%\usetikzlibrary{arrows.meta, calc}
%\usepackage{pgfplots}
%\pgfplotsset{compat=1.11}
%\pgfplotsset{width=95mm,compat=1.10}
%\usepackage{wrapfig}
%\usepackage{cite}

%\let\oldcite=\cite
%\def\cite#1{%
%\addtocounter{page}{-1}%
%\oldcite{#1}%
%\addtocounter{page}{1}%
%}

\def\citepunct{,\,}
\def\prostor{}
%\input glyphtounicode
%\pdfgentounicode=1
%\usepackage{auto-pst-pdf}

%\usepackage{makeidx}
%\makeindex
%\usepackage{multicol}
%\usepackage[flushleft]{threeparttable}
%\usepackage{longtable}

% stříž
%\usepackage{pgfplots,pgfplotstable,animate}
%\pgfplotsset{width=1\textwidth, height=3cm}
%\usepackage{verbatim,wasysym}



\usepackage{listings}
%\ifx\relax

\lstset{
    %inputencoding=utf8,
    extendedchars=true,
    literate=%
    {á}{{\'a}}1
    {č}{{\v{c}}}1
    {ď}{{\v{d}}}1
    {é}{{\'e}}1
    {ě}{{\v{e}}}1
    {í}{{\'i}}1
    {ň}{{\v{n}}}1
    {ó}{{\'o}}1
    {ř}{{\v{r}}}1
    {š}{{\v{s}}}1
    {ť}{{\v{t}}}1
    {ú}{{\'u}}1
    {ů}{{\r{u}}}1
    {ý}{{\'y}}1
    {ž}{{\v{z}}}1
    {Á}{{\'A}}1
    {Č}{{\v{C}}}1
    {Ď}{{\v{D}}}1
    {É}{{\'E}}1
    {Ě}{{\v{E}}}1
    {Í}{{\'I}}1
    {Ň}{{\v{N}}}1
    {Ó}{{\'O}}1
    {Ř}{{\v{R}}}1
    {Š}{{\v{S}}}1
    {Ť}{{\v{T}}}1
    {Ú}{{\'U}}1
    {Ů}{{\r{U}}}1
    {Ý}{{\'Y}}1
    {Ž}{{\v{Z}}}1
}

\lstset{
  %extendedchars=true,
  %inputencoding=utf8,
  breakatwhitespace=true,
  breaklines=true, 
  basicstyle=\ttfamily\footnotesize,
  columns=fullflexible,
  showstringspaces=false,
  commentstyle=\color{gray}\upshape,
  keepspaces=true,
}

%\lstdefinelanguage{XML}
%{
%  morestring=[b]",
%  morestring=[s]{>}{<},
%  morecomment=[s]{<?}{?>},
%  stringstyle=\color{black},
%  identifierstyle=\color{darkblue},
%  keywordstyle=\color{cyan},
%  morekeywords={xmlns,version,type}% list your attributes here
%}
%\fi


\headheight=12.5pt
\headsep=17pt % 19pt
\usepackage{fancyhdr}
\pagestyle{fancy}
\fancyhf{}
\fancyfoot[LE,RO]{\thepage}
\fancyheadoffset[LE,RO]{0pt}
% \marginparsep+\marginparwidth
\renewcommand{\headrulewidth}{0.4pt}
\renewcommand{\footrulewidth}{0.4pt}

\def\pracovnizahlavi{ {\color{red}\bf pracovní verze}}
\gdef\malzahlavitemp{\mbox{Informa\v{c}ní bulletin \v{C}eské statistické spole\v{c}nosti, }}
\fancyhead[RO]{\let\malzahl=\malzahlavitemp
  \malzahl
  %!\the\cislo
  \cislalong
  /\pgfmathparse{int(\rocnik+2008)}\pgfmathresult
\ifnum\zahlavi=1
  \pracovnizahlavi\relax
\else
  \relax
\fi
%\ \pracovnizahlavi % NEW!
}
%\fancyfoot[RE]{\ifnum\zahlavi=0\pracovnizahlavi\fi}
\fancyhead[LE]{\dodatek}

\def\formatujosnovu#1{\textbf{#1}\smallskip\par}
\def\zasah#1{%
  \def\dodatek{#1}%
  \write\maosnova{\formatujosnovu{#1}}%
  \write\maosnovab{$<$tr$>$$<$th colspan=$"$4$"$$>$#1$<$/th$>$$<$/tr$>$\string\odskoc}%
  }

\def\vedecke{\zasah{Vědecké a odborné články}} 
\def\prehledove{\zasah{Přehledové články}} 
\def\odborne{\zasah{Odborné články}} 
\def\jine{\zasah{Jiné statě}} 
% Vědecká, přehledová, jiná stať
\def\zpravy{\zasah{Zprávy a informace}}
\def\pozvanky{\zasah{Pozvánky na akce}}
\addtolength{\voffset}{4.5mm}
%\def\zvyrazni#1{\textit{#1}}

\makeatletter
\renewcommand\@seccntformat[1]{\@nameuse{the#1}\hspace{.5em}} %% byla tečka před \hspace
\makeatother

\def\bs{$\backslash$}
\usepackage{csts}
\widowpenalty=10000
\clubpenalty =10000

\ifnum\zahlavi=0
\overfullrule=0cm % 0cm
\else
\overfullrule=3cm
\fi

\def\z{\discretionary{-}{-}{-}}
\def\odskoc{\string\odskoc\space }
 %  impli-ca-tio-nal

%\DeclareRobustCommand\CS[1]
% {\def\next{#1}\leavevmode{$\cal C$}\kern-.3em
%  \lower .67ex\hbox{$\cal S$}\if E\next\kern-.045em\else\kern-.145em\fi#1}
%\newcommand\zapomen[1]{}
\usepackage{caption}
\captionsetup{justification=centerlast}% bylo justified, centering
%\captionsetup{tableposition=above}
%\captionsetup[table]{position=top}
%\captionsetup{type=table,position=above,skip=3pt}
%\usepackage{enumerate}
%\usepackage{bbm}


% Bohdalová
%\newtheorem{defn}{Definition}[section]
%\newtheorem{example}{Example}[section]

% Klicnarová
%\newtheorem{thm}{Theorem}[section]
%\newtheorem{theorem}{Theorem}[section]

% Friesl
%\usepackage{enumerate}
%\usepackage{amsmath}
%\newtheorem{theorem}{Tvrzení}[section]
%\newtheorem{theoremSK}{Tvrdenie}[section]
%\newtheorem{defi}{Definition}[section]
%\newtheorem{definition}{Definícia}[section]

%\theoremstyle{remark}
%\newtheorem*{remark}{Poznámka}
%\newtheorem*{remarkEN}{Remark}

% Od Ondřeje, až bude potřeba, aktivuj!
% Dodany styl k cislu sekce pridava tecku, takze uprostred cisla
% tvrzeni pak jsou dve tecky za sebou... predefinuju si to:
\makeatletter
%\renewcommand\thetheorem{\@arabic\c@section.\@arabic\c@theorem}
%\renewcommand\thethm{\@arabic\c@section.\@arabic\c@thm}
%\renewcommand\thedefi{\@arabic\c@section.\@arabic\c@defi}
%\renewcommand\thedefn{\@arabic\c@section.\@arabic\c@defn}
%\renewcommand\theexample{\@arabic\c@section.\@arabic\c@example}
%\renewcommand\thedefinition{\@arabic\c@section.\@arabic\c@definition}
%\renewcommand\thetheoremSK{\@arabic\c@section.\@arabic\c@theoremSK}
\makeatother


% ******************  Matematické definice  *******************
%\DeclareMathOperator{\e}{e}
%\newcommand{\RR}{\mathbb R}
%\newcommand{\ZZ}{\mathbb Z}
%\newcommand{\NN}{\mathbb N}
%\newcommand{\TT}{\mathbb T}
%\DeclareMathSymbol{\gi}{\mathbin}{symbols}{"6A}
%\newcommand{\defs}{\mathrel{:=}}
%\DeclareMathOperator{\E}{E}
%\DeclareMathOperator{\PP}{P}
%\newcommand{\tu}{\widetilde u}
%\newcommand{\ta}{\widetilde a}
%\newcommand{\tb}{\widetilde b}
%\newcommand{\tc}{\widetilde c}
%\newcommand{\tX}{\widetilde X}
%\def\dd{\,{\rm d}}
%\def\rmd{{\rm d}}
%\def\argmin{\operatorname{arg\,min}\,}
%\def\argmax{\operatorname{arg\,max}\,}

% Zvára
%\usepackage{natbib}
%\usepackage{Sweave-mal}
%\newcommand{\FKvantil}[3]{\mbox{$F_{#1,#2}(#3)$}}



%\def\der{\mathrm{d}\;\!}
\usepackage{url}
%\renewcommand{\UrlBreaks}{\do\-\do\.\do\_}

\newcommand{\mylink}[1]{\footnote{\texttt{<#1>}}}
\long\def\Abstract#1{%
  {\selectlanguage{english}\frenchspacing\medskip\noindent{\textbf{Abstract:}} #1}\par
  }

\long\def\Abstrakt#1{%
  {\selectlanguage{czech}\medskip\noindent{\textbf{Abstrakt:}} #1}\par
  }

\long\def\AbstraktSK#1{%
  {\selectlanguage{slovak}\medskip\noindent{\textbf{Abstrakt:}} #1}\par
  }

\def\upravalit{%
  \ifx\mujjazyk\macestina \def\refname{Reference}\fi%
  \ifx\mujjazyk\maslovenstina \def\refname{Literatúra}\fi%
  \ifx\mujjazyk\maanglictina \def\refname{References}\fi%
  }

\def\KeyWords#1{\par\noindent{\selectlanguage{english}\frenchspacing\textbf{Keywords:} #1.}\par\upravalit}
\def\KlicovaSlova#1{\par\noindent{\selectlanguage{czech}\textbf{Klíčová slova:} #1.}\par\upravalit}
\def\KlucovaSlova#1{\par\noindent{\selectlanguage{slovak}\textbf{Kľúčová slová:} #1.}\par\upravalit}

\partopsep=-4pt plus 1pt minus 1pt
\ifx
\def\cl{
\itemsep=2.5pt
\parskip=-1.5pt
%\partopsep=-6pt
%\topsep=-6pt
%\parsep=-60pt
}
\fi
\def\engl{\selectlanguage{english}
   \frenchspacing}
%\def\ee{{\rm e}}   
%\def\LGD{\mathop{\rm  LGD}}
%\def\iid{{\rm i.i.d.}}
%\def\var{\mathop{\rm var}}
%\def\cov{\mathop{\rm cov}}
%\def\RVC{\mathop{\rm RVC}\nolimits}
%\def\MDA{\mathop{\rm MDA}\nolimits}
%\def\Poi{\mathop{\rm Poi}\nolimits}
%\def\PML{\mathop{\rm PML}\nolimits}


%\def\mleft{\textrm{\textlangle\;\!}}
%\def\mright{\textrm{\;\!\textrangle}}
%\usepackage[normalem]{ulem}
%\ULdepth=1.5pt
%\let\href=\url

\ifx
\def\hvezdicka{*}
\def\citet#1#2{%
  \def\parameter{#1}%
  \ifx\hvezdicka\parameter
    \cite{#2}%
  \else
    \cite{#1}#2%
  \fi}
\fi

%\usepackage{eurosym}
%\let\texteuro=\officialeuro

% Skopal
%\definecolor{gray}{rgb}{0.4,0.4,0.4}
%\definecolor{darkblue}{rgb}{0.0,0.0,0.6}
%\definecolor{cyan}{rgb}{0.0,0.6,0.6}

% zde BYL listings

% Hasilova
%\usepackage{subfigure}
%\newtheorem{priklad}{Příklad}
%\newtheorem{res}{Řešení}
%\usepackage{bm}
%\usepackage{fancyvrb}

%\newdimen\zde
%\zde=\textwidth
%\advance\zde by -2\fboxrule

\selectlanguage{czech}
\hyphenation{MOVISS ROBUST}

\selectlanguage{english}
\hyphenation{Wil-co-xon}

%\def\malE{\mathsf{E}\;\!}
%\def\velkeE{\mathsf{E}}

% 2020-03-06, malipivo, tři řádky
\renewcommand\rmdefault{cmr}
\renewcommand\sfdefault{cmss}
\renewcommand\ttdefault{cmtt}

% 2020-04-21, malipivo
\newcount\cislo
\newcount\rocnik
\IfFileExists{cislo.tex}
  {\input cislo.tex%\zahlavi=1
  }{%\zahlavi=0
  %\cislo=2 % 1 nebo 3
  }
%\cislo=3
%\pgfmathparse{int(\the\rocnik+2008-2000)}
%\let\mezivypa=\pgfmathresult
\pgfmathparse{int(\the\rocnik+2008)}
\let\mezivypb=\pgfmathresult

%\usepackage{backref}
\usepackage[
  %unicode,
  backref=page,
  %pagebackref,
  citecolor=blue,
  colorlinks,
  %draft,
  ]{hyperref}
\renewcommand*{\backrefalt}[4]{\it\small cit.~#2}
\def\backreftwosep{, }
\def\backreflastsep{ a~}
%\def\backrefpagesname{cit.}
%\renewcommand

\def\malretezec{ne}
%\IfFileExists{prace.tex}{\input prace.tex}{}
\ifx\malretezec\barevnost \hypersetup{colorlinks=false} \fi

\hypersetup{
  hypertexnames=false, bookmarksnumbered, bookmarksopen, bookmarksopenlevel=3,
  %pdftitle={Information Bulletin of the Czech Statistical Society, \the\cislo/\mezivypb},
  %pdfsubject={Published quaterly},
  %pdfcreator={Lua, LuaLaTeX, CSfonts},
  %pdfauthor={Czech Statistical Socienty},
  %pdfkeywords={Statistics, Mathematics, Typography},
  pdftitle={\the\cislo/\mezivypb, Informační bulletin České statistické společnosti},
  pdfsubject={Čtvrtletník společnosti},
  pdfcreator={Lua, LuaHBTeX, CSfonts},
  pdfauthor={Česká statistická společnost},
  pdfkeywords={matematika, statistika, typografie},
}

\ifx\malprace\pracex \else
\hypersetup{colorlinks}
\fi

%\usepackage{bookmark} % [numbered]
\renewcommand\thesection{\arabic{section}.}
\renewcommand\thesubsection{\arabic{section}.\arabic{subsection}.}
\renewcommand\thesubsubsection{\arabic{section}.\arabic{subsection}.\arabic{subsubsection}.}



%\immediate\write18{/bin/ls} %; cp -f maosnova.tex maosnova2.tex

\begin{document}

\selectlanguage{czech}
\hyphenation{log-nor-mál-ní log-nor-mál-ním Bes-ky-dech dal-ších}
\shorthandoff{-}
%\lefthyphenmin=2
%\righthyphenmin=2

\def\mujclear{
  \setcounter{part}{0}
  %\setcounter{chapter}{0}
  \setcounter{section}{0}
  \setcounter{subsection}{0}
  \setcounter{subsubsection}{0}
  \setcounter{paragraph}{0}
  \setcounter{subparagraph}{0}
  %\setcounter{page}{0}
  \setcounter{equation}{0}
  \setcounter{figure}{0}
  \setcounter{table}{0}
  \setcounter{footnote}{0}
  \setcounter{mpfootnote}{0}
  %\setcounter{enumi}{0}
  %\setcounter{enumii}{0}
  %\setcounter{enumiii}{0}
  %\setcounter{enumiv}{0}
  }

\def\vlozstranu{\clearpage}
\def\macestina{czech}
\def\maslovenstina{slovak}
\def\maanglictina{english}


\def\bonushyper{%
\phantomsection
%\bookmark[dest=tocpage, level=0, bold, color=blue]{\mujnazevCS}%
%\phantomsection
%\bookmark[dest=tocpage, level=0, italic, color=green]{\mujauthor}%
%\raisebox{3mm}[0pt][0pt]{%
\addcontentsline{toc}{part}{\mujauthor: \mujnazevCS}%
%}%
%\addtocounter{part}{-1}
%\addcontentsline{toc}{part}{\mujauthor}
%\addtocounter{part}{-1}
}
%\let\oldauthor=\author
%\def\author#1{\oldauthor{#1}\bonushyper}

\def\vloz#1#2{%
  %\global\advance\citachyper by 1
  %\def\stylhyper{\the\citachyper}
  %\renewcommand*{\HyperDestNameFilter}[1]{\the\citachyper-#1}
  \vlozstranu
  \mujclear
\def\ulozsi{#1}%
\begingroup
%\addtocounter{page}{1}%
\phantomsection
%\raisebox{3mm}[0pt][0pt]{%
\label{#1}%
%}%
%\addtocounter{page}{-1}%
\selectlanguage{#2}%
  \frenchspacing % i pro angličtinu, jak je to nezvyk pro našinec
  \def\mujjazyk{#2}%
%  \iflanguage{czech}{\def\refname{Literatura}}{}%
%\ifnum\language=16\gdef\refname{Literatura}\fi%
  \input{zdrojaky/#1}\par\relax%
\endgroup
  %\par\mujnazevEN
  \gdef\vlozstranu{\clearpage}
  %\closeme % !experiment
  %\newpage% !experiment
  }

\def\extras{%
% začátek experimentu
%\closeout\maosnova
%\closeout\maosnovab
%\newwrite\maosnova
%\openout\maosnova=maosnova.tex
%\newwrite\maosnovab
%\openout\maosnovab=maosnovab.tex
% konec experimentu
  \write\maosnova{\emph{\mujauthor}\string\odskoc\space\mujnazevCS\string\pr{\ulozsi}}%
  \write\maosnovab{$<$tr$>$\string\odskoc$<$th$>$\mujauthor$<$/th$>$\string\odskoc\space$<$td$>$\mujnazevPR$<$/td$>$\string\odskoc\space$<$td$>$\mujnazevDR$<$/td$>$\string\pr{\ulozsi}}%
  }

\ifx
\def\notes#1#2#3{%
\begingroup
\def\thefootnote{}
\footnotetext{\kern-1.8em Doručeno redakci: #1, imprimatur: #2.\\
MSC2010: #3, DOI: \textsf{10.5300/IB/2011-2/\thepage}}%
\endgroup
}
\def\medoi#1{\textsf{doi:\,#1}}
\fi

\newcommand{\nazevEN}[1]{
    \noindent
    \raggedright
    \large\textbf{\MakeUppercase{#1}}
    \medskip
    }
\newcommand{\nazevCS}[1]{
    \noindent
    \raggedright
    \large\textbf{\MakeUppercase{#1}}
    \medskip
    }

\makeatletter
\def\dotpfill{%
  \leavevmode%
  \leaders\hb@xt@.33em{\hss.\hss}\hfill%
  \kern\z@}
\makeatother 
\newcommand{\pr}[1]{\,\dotpfill\makebox[4.5mm][r]{\pageref{#1}}\\[4pt]}



\newwrite\maosnova
\openout\maosnova=maosnova.tex

\newwrite\maosnovab
\openout\maosnovab=maosnovab.tex
\write\maosnovab{$<$table border=$"$1$"$$>$\string\odskoc\string\prostor\string\odskoc}

\def\mastrana{%
  \def\vlozstranu{}%\vfill
  }

\def\closeme{%
\newwrite\stran
\immediate\openout\stran=stran.tex
\immediate\write\stran{\stran=\thepage}
\closeout\stran
%\write\maosnovab{\string\odskoc$<$/table$>$\string\odskoc\string\prostor}
}


\ifx\relax
\IfFileExists{maosnova2.tex}
  {{\parindent=0pt\textbf{Pracovní osnova\ldots}\medskip\par\input{maosnova2.tex}}}
  {Zde bude obsah\ldots}
\fi




\newcount\hodnotaA
\newcount\hodnotaB

\write\maosnovab{$<$tr$>$\string\odskoc$<$th colspan=$"$4$"$$>$\string\odskoc}
\write\maosnovab{$<$a name=$"$\string\the\hodnotaA-\string\the\cislo$"$$>$\string\the\cislo/\string\the\hodnotaB$<$/a$>$\string\odskoc}
\write\maosnovab{$<$a title=$"$http://statspol.cz/bulletiny/ib-\string\the\hodnotaB-\string\the\cislo-web.pdf$"$ href=$"$ib-\string\the\hodnotaB-\string\the\cislo-web.pdf$"$$>$Bulletin v pdf$<$/a$>$\string\odskoc$<$/tr$>$\string\odskoc}

%\renewcommand\arraystretch{1.25}
%\allowdisplaybreaks
%\fboxsep=0pt
%\def\logit{\mathop{\rm logit}}
%\def\malee{{\rm e}}

%\def\mujnazevCS{}
%\def\mujnazevEN{}
%\def\mujnazevPR{}
%\def\mujnazevDR{}
%\def\mujauthor{}
\graphicspath{
   {zdrojaky/striz-manim/obr/}
   {zdrojaky/striz-novinky-r/obr/}
   {zdrojaky/striz-sum/obr/}
   {zdrojaky/striz-sum/tikz-vypln/}
   {zdrojaky/striz-novinky-tex/obr/}
   {zdrojaky/robust-covid-19/}
   }
\def\TeX{T\kern-.1667em\lower.5ex\hbox{E}\kern-.125emX} % bylo na \protect\TeX ???

\ifx\malprace\pracex
%, přední strana obálky
\newpage
\thispagestyle{empty}
\begin{tikzpicture}[remember picture, overlay]
\node at (current page) {\IfFileExists{obalka/obalka.pdf}{\includegraphics[viewport=449 27.5 870.263 623.691, width=\paperwidth, height=\paperheight]{obalka/obalka.pdf}}{\relax}};
\end{tikzpicture}
%\addtocounter{page}{-1}

% zadní strana obálky
\newpage
\thispagestyle{empty}
\begin{tikzpicture}[remember picture, overlay]
\node at (current page) {\IfFileExists{obalka/obalka.pdf}{\includegraphics[viewport=28.371 27.5 449.25 623.691, width=\paperwidth, height=\paperheight]{obalka/obalka.pdf}}{\relax}};
\node[anchor=north west, xshift=17mm, yshift=-11mm] at (current page.north west) {
\Large
\begin{minipage}{16.8cm}
\textbf{Obsah}\\[3pt]
\def\odskoc{\\}
\IfFileExists{maosnova2}{\input maosnova2.tex}{\relax}
\end{minipage}};
\end{tikzpicture}
\fi % \malprace versus \pracex

\def\xbonus{}
\ifnum\cislo=3
%\vedecke
\zpravy
%\pozvanky
\vloz{striz-novinky-r/striz-novinky-r}{czech}                   % 8
\vloz{striz-novinky-tex/striz-novinky-tex}{czech}  % 13
\vloz{robust-covid-19/robust-update.tex}{czech}  %  1
% 22
\newpage test1
\newpage test2
\fi

\ifnum\cislo=4
\zpravy
\vloz{striz-manim/striz-manim}{czech}                 %  9
\vloz{striz-sum/striz-sum}{czech}                             % 7
% 16
\fi

\closeme

\write\maosnovab{\string\odskoc$<$/table$>$\string\odskoc\string\prostor} %\newpage
\closeout\maosnova
\closeout\maosnovab %\closeme

% nefunguje, proč?
%\write18{cp -f maosnova.tex maosnova2.tex} %\newpage

\end{document}

}{\relax}
\makeatother
%\parindent=0pt

\begin{document}
\shorthandoff{-}
%\ifnum\moje=1
%\gdef\dodatek{1}
%\else
%\gdef\dodatek{1,{}}
%\fi

\includepdf[fitpaper,pages={1}, %1,{}
  pagecommand={%
%\begin{picture}(0,0)(0,0)% 
%\put(-87,-50){%
\begin{tikzpicture}[remember picture, overlay, inner sep=0pt, outer sep=0pt]
\node[draw=none, anchor=north west, align=left, xshift=23mm, yshift=-18.5mm, 
] at (current page.north west){%
\ifnum\malosnova=1
\begin{minipage}{11.9cm}%[!t]
\textbf{Obsah}\\[3pt]
%\def\odskoc{} % zmizí vše
\IfFileExists{../maosnova.tex}{\input ../maosnova.tex}{\relax}
\end{minipage}%
\fi
};
%}
%\end{picture}%
\end{tikzpicture}%
%
  \begin{picture}(0,0)(0,0)
    \put(515,-430){\makebox[0pt][c]{\bf\Large Ročník \the\mrocnik, 
%číslo 1--2, 
číslo
\cislalong, 
\mtext
%červen, 2015%
}}
%
    \put(-83.5,-363){%-340
    \begin{minipage}[l]{11.9cm}%
\footnotesize
    \textbf{Informační bulletin České statistické společnosti} vychází čtyřikrát do roka v českém vydání. Příležitostně i mimořádné české a anglické číslo. Vydavatelem je Česká statistická společnost, IČ 00550795, adresa společnosti je %Sokolovská~83, 186\,00 Praha~8. 
Na padesátém~81, 100\,82 Praha~10.
Evidenční 
číslo registrace vedené Ministerstvem kultury ČR dle zákona č.~46/2000 Sb. je E 21214. \\Časopis je sázen v programu \TeX, ve formátu LuaHB\TeX\ s písmy balíku \CSfonts.%\\[6pt]
%Časopis je zařazen do seznamu Rady pro výzkum, vývoj\\a inovace, více viz server \texttt{http://www.vyzkum.cz/}.
%Časopis je na Seznamu recenzovaných neimpaktovaných periodik vydávaných v ČR, více viz server \texttt{http://www.vyzkum.cz/}.
    \\[6pt]
The Information Bulletin of the Czech Statistical Society is published quarterly.\\
%The contribu\-tions in bulletin are published in English, Czech and Slovak languages.
The contribu\-tions in the journal are published in English, Czech and Slovak languages.
    \\[6pt]
    \textbf{Předseda společnosti:} %RNDr. Marek \textsc{Malý}, CSc., Státní zdravotní ústav,  Šrobárova 48, \\ Praha~10, 100\,42, e-mail: \texttt{mmaly@szu.cz}.
    Mgr. Ondřej Vencálek, Ph.D., Katedra matematické analýzy a~aplikací matematiky, Přírodovědecká fakulta Univerzity Palackého, 17. listopadu 12, 771\,46 Olomouc, e-mail: \texttt{ondrej.vencalek@upol.cz}.
    \\[6pt]
    \textbf{Redakce:} % Redakční rada 
prof. RNDr. Gejza \textsc{Dohnal}, CSc. (šéfredaktor), 
prof. RNDr. Jaromír \textsc{Antoch}, CSc., 
%prof. Ing. Václav \textsc{Čermák}, DrSc.,
%doc. Ing. \mbox{Jozef} \textsc{Chajdiak}, CSc.,
doc. RNDr. Zdeněk \textsc{Karpíšek}, CSc., 
RNDr. Marek \textsc{Malý}, CSc., 
doc. RNDr. Jiří \textsc{Michálek}, CSc., 
prof. Ing. Jiří \textsc{Militký}, CSc., 
doc. Ing. Iveta \textsc{Stankovičová}, PhD.,
doc. Ing. Josef \textsc{Tvrdík}, CSc., 
Mgr. Ondřej \textsc{Vencálek}, Ph.D.
    \\[6pt]
    \textbf{Redaktor časopisu:} Mgr. Ondřej \textsc{Vencálek}, Ph.D., \texttt{ondrej.vencalek@upol.cz}.
    \\%[6pt]
    Informace pro autory jsou na stránkách společnosti,  \texttt{http://www.statspol.cz/}.
    \\[6pt]
    \textbf{DOI: 10.5300/IB, http://dx.doi.org/10.5300/IB}\\
    \textbf{ISSN 1210--8022 (Print), ISSN 1804--8617 (Online)}%\\
\\[6pt]
%DOI je přiřazováno ve spolupráci s Čs. sdružením uživatelů \TeX u.\\
%{\color{white}
Toto číslo bylo vytištěno s laskavou podporou Českého statistického úřadu.%}
    \end{minipage}
    }% End of put.
    \linethickness{15pt}
%
\ifnum\moje=0
%
\citaj=0
\loop
\advance\citaj by 1
  \put(301.71,\vyska){\makebox[0pt][c]{{\color{black}
  \rotatebox{45}{\line(0,0){15}}% 301.5 309
  %\circle*{15}%
  }}}
  \advance\vyska by 23 %23 20
\ifnum\citaj<\obdelniku\repeat
\advance\vyska by -2 %-2 -4
%
\citaj=0
\ifnum\ctvercu>0
\loop
\advance\citaj by 1
  \put(301.5,\vyska){\makebox[0pt][c]{\line(0,0){15}}}
  \advance\vyska by 20
\ifnum\citaj<\ctvercu\repeat
\fi % čtvercu>nula?
\advance\vyska by 9
%
\citaj=0
\loop
\advance\citaj by 1
  \put(309,\vyska){\makebox[0pt][c]{\makebox[0pt][c]{{\color{mygray}\circle*{15}}}}}
  \advance\vyska by 20
\ifnum\citaj<\cislo\repeat
%
%
% OLD
\ifx
\loop
\advance\citaj by 1
  \put(301.5,\vyska){\makebox[0pt][c]{\line(0,0){15}}}
  \advance\vyska by 20
\ifnum\citaj<\rocnik\repeat
\advance\vyska by 9
%
\citaj=0
\loop
\advance\citaj by 1
  %\put(307,\vyska){\makebox[0pt][c]{\makebox[0pt][c]{\circle{11}}}}
%  \put(301.5,\vyska){\makebox[0pt][c]{\line(0,0){15}}}
  \put(309,\vyska){\makebox[0pt][c]{\makebox[0pt][c]{{\color{mygray}\circle*{15}}}}}
  \advance\vyska by 20
\ifnum\citaj<\cislo\repeat
\fi
%
\fi
%  
  \end{picture}
  }% End of command.
  ]{IB0-rot}
\end{document}



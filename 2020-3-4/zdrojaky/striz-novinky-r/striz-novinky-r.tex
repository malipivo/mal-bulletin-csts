% !TEX TS-program = LuaLaTeX
% !TEX encoding = UTF-8 Unicode 
% !TEX root = ../../mal-core.tex

\def\malI{\color{blue}{\textbf{i}}}
\def\malE{\color{green}{\tiny\CheckmarkBold}} % \checkmark

\gdef\mujnazevCS{Novinky ze světa R+koronaviru}
\gdef\mujnazevEN{News from the world of R+coronavirus}
\gdef\mujnazevPR{\mujnazevCS}
\gdef\mujnazevDR{\mujnazevEN}
\gdef\mujauthor{Pavel Stříž}

\bonushyper

\nazev{\mujnazevCS}

\nazev{\mujnazevEN}

\author{\mujauthor}

\Email{pavel@striz.cz}
\medskip

\noindent
\footnotesize
\textit{The R Foundation Retweeted: Peter Dalgaard.
R 4.0.0 ``Arbor Day"\,\,(source version) has been released.}\smallskip
\normalsize
  
\noindent
24.\,4.\,2020 mi přistála na stole 
\href{https://twitter.com/pdalgd/status/1253585277949489152}{tato zpráva} 
a za pár dní na to, 28.\,4.\,2020, došlo k \href{https://bioconductor.org/news/bioc_3_11_release/}{aktualizaci} Bioconductoru na verzi 3.11. Je Svátek práce, jdu ty nové \texttt{lazyLoad}, \texttt{lazyEval} a \texttt{lazyData} v R vyzkoušet a sdělit svůj nezávislý pohled.

\section{R v4.0.0}

Bez otálení jsem na svém Xubuntu 18.04 do \texttt{/etc/apt/sources.list} přidal:
\begin{lstlisting}
deb https://cloud.r-project.org/bin/linux/ubuntu bionic-cran40/
\end{lstlisting}

Zakomentoval jsem starší pokusy a provedl preventivní kroky:
\begin{lstlisting}
$ sudo apt-key adv --keyserver keyserver.ubuntu.com --recv-keys E298A3A825C0D65DFD57CBB651716619E084DAB9
$ sudo apt update
$ sudo apt upgrade
\end{lstlisting}

Jádro aktualizace pak tvořil příkaz:
\begin{lstlisting}
$ sudo apt install r-base r-base-dev
\end{lstlisting}

Rychlý test prokázal, že se podařilo a provedl jsem aktualizaci knihoven:
\begin{lstlisting}
$ R --version
$ R
> update.packages(.libPaths())
\end{lstlisting}


\section{COVID v19}

\noindent
\footnotesize
\textit{The R Foundation Retweeted: R Consortium has started new GitHub repository to centralize collaboration and data sources -- looking to develop COVID-19 tools and code -- Come add your information and contribute to the community! \url{https://bddy.me/3aAX0mb}}\smallskip
\normalsize

\noindent
Z toho samého dne zaujala mou pozornost ještě \href{https://twitter.com/RConsortium/status/1253715512426663936}{tato zpráva.} Řekl jsem si, že bych měl nově nainstalované R hlouběji vyzkoušet.

Na úvodní stránce na mne vyskočily 4 projekty: 
    \href{https://github.com/RConsortium/r-collaboration/blob/master/projects/covid-19/covid-19-tracker.md}{Coronavirus Tracker,}
    \href{https://github.com/RConsortium/r-collaboration/blob/master/projects/covid-19/covid-19-propagation.md}{COVID-19 Propagation,} 
    \href{https://github.com/RConsortium/r-collaboration/blob/master/projects/covid-19/covid-19-tracker-map.md}{COVID-19 Tracker Map} a
    \href{https://github.com/volzinnovation/covid-19_SARS-CoV-2_corona}{COVID-19 Projections.}

Vezmu-li to od konce. U \href{https://github.com/volzinnovation/covid-19_SARS-CoV-2_corona}{Projections} na mne vyskočil GitHub. Po určitém bádání se mi podařilo otevřít \href{https://www.volzinnovation.com/covid-19_SARS-CoV-2_corona/}{rozcestník} a webovou stránku pro Českou republiku, \href{https://www.volzinnovation.com/covid-19_SARS-CoV-2_corona/reports/latest/Czechia.html}{\url{volzinnovation.com/covid-19_SARS-CoV-2_corona/reports/latest/Czechia.html}.} Autorem je Raphael Volz.

Autorem \href{https://github.com/RConsortium/r-collaboration/blob/master/projects/covid-19/covid-19-tracker-map.md}{Tracker Map} je Jay Ulfelder. V \href{https://rstudio.cloud/project/1013588}{RStudiu Cloud} mne hned upozornili, že se jedná o dočasný projekt. Pod \href{https://dartthrowingchimp.shinyapps.io/covid19-app/}{Shiny app} na mne vedle analýz vyškočily pdf v záložce WHO Situation Reports. Zajímavý nápad.

Druhý projekt v pořadí je \href{https://github.com/RConsortium/r-collaboration/blob/master/projects/covid-19/covid-19-propagation.md}{Propagation.} Autorem je Juan Francisco Venegas Gutiérrez. Bohužel repozitář na GitHubu mi nešel otevřít, tak jsem to zahlásil (Issues). Mezi modely jsem Českou republiku nenašel.

\section{Coronavirus Tracker v0.1.4}

První projekt v pořadí od Johna Coeneho \href{https://github.com/RConsortium/r-collaboration/blob/master/projects/covid-19/covid-19-tracker.md}{Coronavirus Tracker} vyzývá ke spuštění R. Pustil jsem se do toho. \href{https://rstudio.cloud/project/950251}{RStudio cloud} občas jel bez přístupových práv, požadavek na knihovny 
\href{https://rinterface.github.io/shinyMobile/}{\texttt{shinyMobile}} a 
\href{https://echarts4r.john-coene.com/}{\texttt{echarts4r}}
zněl zajímavě.

\begin{lstlisting}
$ R
> install.packages("coronavirus")
\end{lstlisting}

Zkusil jsem z dokumentace první ukázku a ještě si vyžádal knihovnu 
\href{https://cran.r-project.org/web/packages/dplyr/index.html}{\texttt{dplyr}.} Proč si ji nenainstaloval sám?
\begin{lstlisting}
> install.packages("dplyr")
> library(coronavirus)
> require(dplyr)
> coronavirus %>% filter(type=="confirmed") %>% group_by(Country.Region) %>% summarise(total=sum(cases)) %>% arrange(-total) %>% head(20)
# A tibble: 20 x 2
   Country.Region       total
   <chr>                <int>
 1 Mainland China       70446
 2 Others                 355
 3 Singapore               75
\end{lstlisting}

Tady jsem zbystřil. To jsou stará data v textové formě, nikoliv hezké mapy přes web s aktuálními daty. 
Ve slangové řeči: rtfm! 
To, co jsem právě nainstaloval, je knihovna \href{https://github.com/RamiKrispin/coronavirus}{\url{https://github.com/RamiKrispin/coronavirus},} která má stejný název. Mezi Issues jsem autorovi Trackeru zahlásil, že jeho název je v konfliktu \href{https://cran.r-project.org/web/packages/coronavirus/index.html}{s existující knihovnou} z února 2020.

Pokračoval jsem v experimentování, knihovnu jsem odinstaloval a podíval se na návod, \href{https://coronavirus.john-coene.com/#/get-started}{\url{https://coronavirus.john-coene.com}.}
\begin{lstlisting}
> remove.packages("coronavirus")
> install.packages("remotes")
> remotes::install_github("Johncoene/coronavirus")
\end{lstlisting}

Během instalace mi naskočila tato neobvyklá zpráva:
\begin{lstlisting}
- Use `usethis::browse_github_pat()` to create a Personal Access Token.
- Use `usethis::edit_r_environ()` and add the token as `GITHUB_PAT`.
\end{lstlisting}

Knihovna \texttt{usethis} se teprve instalovala. V každém případě zmínka o~\texttt{Rate limit reset at: [...]}, kdy došlo k restartu za několik minut pro mne znamenalo chvíli počkat a instalaci zopakovat.

Před koncem instalace si R vyžádalo systémový balík \texttt{libpq-dev} ve starší verzi 10.3-1 (aktuální je 10.12-0). Udělal jsem hrubý krok, doporučuji čtenářům najít lepší řešení přes Docker či pečlivě projít, co se bude odinstalovávat.

\begin{lstlisting}
$ sudo apt install aptitude
$ sudo aptitude install libpq-dev
\end{lstlisting}

Na první dotaz jsem dal nikoliv (\texttt{n}; starší balík by se nenainstaloval), na druhou nabídku ano (\texttt{y}). Zopakoval jsem instalaci v R a skončila úspěšně.

Zkusil jsem třířádkovou ukázku dle instalačního manuálu, \href{https://coronavirus.john-coene.com}{\url{coronavirus.john-coene.com}} (nutno zalistovat na webové stránce níž). 

\begin{lstlisting}[escapechar=!]
> library(coronavirus)
> virus<-crawl_coronavirus()
!\malI! Crawling data from John Hopkins
!\malI! Crawling data from Weixin
!\malI! Crawling data from DXY
> run_app(virus)
\end{lstlisting}

Pozn. Pokud bychom se dostali do konfliktu u příkazů, užijme:
\begin{lstlisting}
> virus <- coronavirus::crawl_coronavirus()
> coronavirus::run_app(virus)
\end{lstlisting}

Ve webovém prohlížeči se mi otevřela vygenerovaná stránka, pokaždé na jiném portu. Radost byla veliká! Za pozornost stojí, že má být John\underline{s} Hopkins, to již někdo zahlásil autorovi k opravení.

\begin{figure}[!hbt]
\hfil
\includegraphics[width=0.85\textwidth]{nahled-jhu.png}
\end{figure}

Ve spodní části je pět záložek. Na druhé (waveform\_path) v bloku \textit{China} a \textit{World} a čtvrté (graph\_circle) v bloku \textit{Cities} jako kdyby něco chybělo. Po nakliknutí do bloku se otevře detailní výpis. Vrátit zpět se dá přes značku xmark\_circle\_fill v pravém horním rohu. Design je trochu nezvyklý, ale musíme mít na paměti, že je to zaměřené na mobilní telefony a já to zkoušel na notebooku.

Detaily kolem dat je možné nalézt v levém horním rohu pod bars nebo menu. Z R lze server zavřít přes klávesovou kombinaci Ctrl+C.

Nyní dokumentace radí si nastavit \texttt{crontab} atd. Co mne zaujalo u stažení dat z DXY je, že se občas nezadařilo připojit. V rychlosti jsem nahlédl na server \href{https://education.rstudio.com/}{\url{https://education.rstudio.com/},} konkrétně na \href{https://rstudio-education.github.io/hopr/dataio.html}{\url{dataio}.}

Jakmile se podařilo na servery připojit, mohl jsem si proměnnou \texttt{virus} uložit a opětovně užívat. Rychlá pomůcka u experimentů bez nutnosti aktualizace dat.

\begin{lstlisting}
> save(virus, file="virus.RData")
> load("virus.RData")
\end{lstlisting}


\section{Hašovací klíč od \texttt{newsapi.org} v2}

Mou pozornost zaujala poslední záložka se zprávou \texttt{No newsapi token.} To bych rád poléčil. Autor v \href{https://coronavirus.john-coene.com/#/prod}{dokumentaci} radí:

\begin{lstlisting}
> library(coronavirus)
> create_config()
\end{lstlisting}

V pozadí se ze šablony vytvoří soubor \texttt{\_coronavirus.yml}, blok \texttt{database} je povinný, blok \texttt{newsapi} volitelný. To byl pro mne problém. Já jsem to chtěl obráceně. Nevadí.

Přes \url{https://newsapi.org/register} jsem se zaregistroval a získal hašovací klíč. Zahlédl jsem jejich novou knihovnu pro R \href{https://cran.r-project.org/web/packages/newsanchor/index.html}{\texttt{newsanchor},} my zůstaneme u autorem užité knihovny \href{https://github.com/news-r/newsapi}{\texttt{newsapi}.} 

Zkusil jsem R podsunout hašovací klíč:
\begin{lstlisting}
> library(newsapi)
> newsapi::newsapi_key("41e22e9efcf64b2a9354a796b99c43b8")
\end{lstlisting}

Ale ani touto cestou ani jinou přes editaci souboru \texttt{\_coronavirus.yml} se mi to nepodařilo.

Prvně jsem nahlédl na zdrojové kódy v:
\begin{lstlisting}
$ cd ~/R/x86_64-pc-linux-gnu-library/4.0/coronavirus
\end{lstlisting}

Narazil jsem hlavně na binární soubory rds, rdx a rdb. Nejsem expert, abych dokázal odpovědět, jestli by se soubory daly rozluštit a editovat.

Prozkoumal jsem zdrojové kódy přímo od autora:
\begin{lstlisting}
$ git clone https://github.com/JohnCoene/coronavirus
\end{lstlisting}

Došel jsem k závěru, že bych musel zdrojové kódy upravit, zkompilovat atd. To je nad rámec této sváteční zprávy. 

V souboru \texttt{coronavirus/inst/app/Dockerfile} jsem si ověřil, že skutečně knihovnu 
\href{https://github.com/news-r/newsapi}{\texttt{newsapi}} přebírá z GitHubu od uživatele 
\href{https://github.com/news-r}{\texttt{news-r}.}




\section{PostgreSQL v10+190}

Říkal jsem si, když už se mi podařilo nainstalovat \texttt{libpq-dev}, dokáži i zbytek. Otevřel jsem \href{https://www.digitalocean.com/community/tutorials/how-to-install-and-use-postgresql-on-ubuntu-18-04}{komunitní tutoriál.} Nainstaloval jsem PostgreSQL:
\begin{lstlisting}
$ sudo apt install postgresql postgresql-contrib
\end{lstlisting}

A nejkratší možnou cestou jsem se pustil do dalších kroků. Vytvořil jsem v databázovém systému nového uživatele \texttt{testing} a novou databázi \texttt{testing}. Vynechávám krok vytvoření uživatele pod operačním systémem.
\begin{lstlisting}
$ sudo -i -u postgres createuser --interactive
Enter name of role to add: testing
Shall the new role be a superuser? (y/n) y
$ sudo -u postgres createdb testing
\end{lstlisting}

Uživatel je bez hesla, to webové rozhraní nepřijme. Nastavil jsem nové heslo přes:
\begin{lstlisting}
$ sudo -i -u postgres
$ psql
postgres=# ALTER USER testing WITH PASSWORD 'testing';
postgres=# \q
$ exit
\end{lstlisting}

Rychlokurz \texttt{psql}: \texttt{help} je základní nápověda, \textbackslash\texttt{l} je výpis databází, \textbackslash\texttt{c testing} je připojení k naší databázi, \textbackslash\texttt{dt} je výpis tabulek, \textbackslash\texttt{h} je seznam SQL příkazů, \textbackslash\texttt{?} je seznam příkazů \texttt{psql} a \textbackslash\texttt{q} ukončí běh programu. Verzálky u~příkazů netřeba psát.


Vše zrealizované jsem zaznačil v \texttt{\_coronavirus.yml}:
\begin{lstlisting}
database:
  name: testing
  host: 127.0.0.1
  user: testing
  password: testing
newsapi:
  key: 41e22e9efcf64b2a9354a796b99c43b8
\end{lstlisting}
% záloha: user: malipivopostgres heslo: postgresmalipivo

Když jsem opakoval tři řádky ukázkového spuštění v R, výpis se mi rozšířil o tyto dva řádky:
\begin{lstlisting}[escapechar=!]
!\malI! Crawling news from newsapi.org
!\malE! Writing to database
\end{lstlisting}
%✔(zelená)


V páté záložce mi vyběhly novinky, aktivovaný dotaz lze nalézt u autora v souboru \texttt{coronavirus/R/crawl.R}:
\begin{lstlisting}
news <- newsapi::every_news("coronavirus OR covid", results = 100, language = "en", sort = "popularity")
\end{lstlisting}

\begin{figure}[!htb]
\includegraphics[width=\textwidth]{nahled-news.png}
\end{figure}

Ověření funkčnosti můžeme zjistit i z tabulky \texttt{log}:
\begin{lstlisting}
$ psql -h localhost -d testing -U testing
Password for user testing: testing
testing=# SELECT * FROM log;
\end{lstlisting}

%\newpage
Dostáváme přibližně takový výsledek:
\begin{lstlisting}
         last_updated
-------------------------------
 2020-05-01 18:19:24.547171+02
(1 row)
\end{lstlisting}

Dle chuti lze dál bádat u surových dat, např.:
\begin{lstlisting}
testing=# SELECT * FROM jhu WHERE country='Czechia';
testing=# SELECT * FROM jhu WHERE country='Slovakia';
testing=# \q
\end{lstlisting}



\section{Bioconductor v3.11}

Před dalším krokem si obvykle nastavuji plná práva u těchto adresářů:
\begin{lstlisting}
$ cd /usr/lib/R
$ sudo chmod -R 777 site-library/
$ sudo chmod -R 777 library/
$ cd /usr/share
$ sudo chmod -R 777 R/
\end{lstlisting}

Za zmínku stojí, že manažer knihoven \texttt{biocLite} ustupuje a roli nahrazuje \href{https://www.bioconductor.org/install/}{\texttt{BiocManager}.} V R lze otestovat:
\begin{lstlisting}
> install.packages("BiocManager")
> library(BiocManager)
> BiocManager::install()
> BiocManager::available()
\end{lstlisting}

Můžeme ověřit instalaci knihoven:

\begin{lstlisting}
> BiocManager::valid()
   [...] "coronavirus", "echarts4r", "shinyMobile" [...]
Warning message:
0 packages out-of-date; 3 packages too new 
\end{lstlisting}

To souhlasí, neb \texttt{coronavirus} byl instalován z 
\href{https://github.com/JohnCoene/coronavirus}{GitHubu,} nikoliv z \href{https://cran.r-project.org/}{CRANu.}

Pro badatele stojí za pozornost obrazy pro \href{https://www.bioconductor.org/help/docker/}{Docker} a \href{https://www.bioconductor.org/help/bioconductor-cloud-ami/}{Amazon} (Amazon Machine Image, AMI). Je zde možnost instalovat vývojářské knihovny:
\begin{lstlisting}
> BiocManager::install(version="devel")
\end{lstlisting}




\section{Řešení konfliktu názvu knihovny v R}

Na chvíli se ještě vraťme k řešení konfliktu stejného názvu knihoven. Na \href{https://stackoverflow.com/questions/52447227/how-do-i-install-an-r-package-under-another-name}{Stackoverflow} zmiňují v principu dvě cesty.

Stáhnout si zdrojové soubory a nic neměnit:
\begin{lstlisting}
$ cd /tmp
$ wget https://cran.r-project.org/src/contrib/coronavirus_0.1.0.tar.gz
$ R CMD INSTALL -l /tmp coronavirus_0.1.0.tar.gz
\end{lstlisting}

%\newpage
V R si pak volat jeden z příkazů a vybrat tak chtěnou knihovnu:

\begin{lstlisting}
> # library(coronavirus)
> library(coronavirus, lib.loc="/tmp")
\end{lstlisting}

Když jsem zkoušel paralelně spustit i instalovanou knihovnu z GitHubu \texttt{coronavirus} odkomentováním prvního řádku, tak to neběželo. Tuším, že se jedná o bezpečnostní pojistku.

Druhá cesta je zasáhnout do souboru \texttt{DESCRIPTION}.
\begin{lstlisting}
$ tar xvf coronavirus_0.1.0.tar.gz
$ mv coronavirus coronavirusRami
$ cd coronavirusRami
$ nano DESCRIPTION
\end{lstlisting}

První řádek upravit například na
%\begin{lstlisting}
\texttt{Package: coronavirusRami}.  
%\texttt{Ctrl+x} 
%\texttt{y}
%\end{lstlisting}

Volitelně upravíme i \texttt{MD5}, konkrétně první řádek za výpis:
\begin{lstlisting}
$ md5sum -b DESCRIPTION
324f8275940bfa7fde376934c57a28ae *DESCRIPTION
\end{lstlisting}

Chtělo by to přejmenovat i další soubory na \texttt{coronavirusRami}, u této školní ukázky vynechávám. Zabalil jsem si zpět a už podsunul R:
\begin{lstlisting}
$ cd ..
$ tar cvf coronavirusRami.tar.gz coronavirusRami/
$ R CMD INSTALL coronavirusRami.tar.gz 
\end{lstlisting}

Ověřit funkčnost můžeme už přímo v R a lze si spustit oba konfliktní balíčky paralelně:
\begin{lstlisting}
> library(coronavirus)
> ?coronavirus::crawl_coronavirus
> library(coronavirusRami)
> ?coronavirusRami::coronavirus
\end{lstlisting}



\section{Pár tipů místo Závěru}

Podobně jako jsou v R knihovny setříděné podle kategorií, viz Zobrazení úloh (\href{https://cran.r-project.org/web/views/}{CRAN Task Views}), lze nahlédnout u 
\href{https://rstudio.com/}{RStudia} na \href{https://rviews.rstudio.com/tags/covid-19/}{R Views} s klíčovým slovem \texttt{covid-19}. K dnešnímu dni tam jsou tři záznamy: 
\href{https://rviews.rstudio.com/2020/04/07/some-select-covid-19-modeling-resources/}{Some Select COVID-19 Modeling Resources,}
\href{https://rviews.rstudio.com/2020/03/19/simulating-covid-19-interventions-with-r/}{Simulating COVID-19 interventions with R} a 
\href{https://rviews.rstudio.com/2020/03/05/covid-19-epidemiology-with-r/}{COVID-19 epidemiology with~R.}

Kdo by dal přednost odpočinku od R a PostgreSQL, nechť nahlédne na aktuální stavy kolem koronaviru na \href{https://www.twitch.tv/killars}{\url{https://www.twitch.tv/killars}.}


% přístup do postgresu zvenčí

% /etc/postgresql/10/main/postgresql.conf
%listen_addresses = '*' 

% /etc/postgresql/10/main/pg_hba.conf
%host    all             all              0.0.0.0/0                       md5
%host    all             all              ::/0                            md5

% $ sudo systemctl restart postgresql

% jak přemluvit run_app v R?
% run_app(virus, embed_url="192.168.1.3")
% nezabralo

